% LaTex document
%\pagelayout{normal}

\def\G#1{{\bf \it #1}}
\documentclass[12pt]{article}
\usepackage{lscape}
\topmargin -0.5in
\oddsidemargin 0in
\evensidemargin 0in
\textheight 8.9in
\textwidth 6.5in
\parskip 3pt plus 1pt minus 0.5pt

\begin{document}
\title{ Continuous Transformations Which Preserve the Structure of a Finite Group}
\author{James N. Bellinger}
\date{11 November 1987}
\maketitle
\begin{abstract}
Some finite groups have a linear mapping
of the group elements onto the algebra with the basis of the group
elements such
that the resulting linear combinations of group elements retain
the group properties.  In some cases these mappings are
continuous Lie groups.  So far I have found
a `non-compact' SO(3)xU(1), SO(4)xU(1), and Sl(2,c)xU(1).
\end{abstract}
%% Kac-Moody?  The generators seem to come in 3's . . .
\newpage
\pagestyle{plain}

\vskip 2in


\section{ Introduction and Physics Motivation}

\par
  Particles such as the $K_L$ we describe as
a combination of $K^0$ and anti-$K^0$.  
Particles such as the $\pi^0$ are regarded as linear combinations of
quarks and anti-quarks.
It has been proposed that
the particles we know or infer (leptons and quarks) are similarly
composed of combinations of ``preons.''

 I examine in this paper some consequences of a simple preon model,
and show that certain continuous symmetries result.

 The model I use has seven basic assumptions:

 $\bullet$  First suppose that all preon interactions are 3-body interactions.
I know of no 4-body interaction which has been observed, now that
beta decay is known to progress through a series of 3-body interactions.
A 4-gluon vertex is hypothesised, but not directly observed.

 $\bullet$  Suppose secondly that all particles are created from combinations of
these preons.  This condition is not required for the mathematics of
the model, but for any possible interpretation of it.

 $\bullet$  Suppose thirdly that we are not for the moment concerned with the
particle positions or momenta, but only with particle type.  We now are
treating all particle interactions as being of the form $A \bigodot B
\rightarrow C$.

 $\bullet$  Fourthly suppose that every particle may interact with every other
to make a new particle.  This requires that the particles in the model
not have any net charge (as in $(e^+ +e^-)/\sqrt{2}$ vs
$(e^+ -e^-)/\sqrt{2}$).  I include no mechanism for giving the particles
mass:  all are assumed massless.


 $\bullet$  Suppose fifthly that the interaction we have defined so far is
associative: $A \bigodot (B \bigodot C) = (A \bigodot B) \bigodot C$.
We can argue that
associativity is a consequence of some of the conservation
laws, though tighter conditions could be applied.  This condition
is deliberately loose.

 $\bullet$  Suppose sixthly that we have some particle $Q$ which has the property
that $Q \bigodot A \rightarrow A$.  We do in fact observe interactions
with this property, such as an electron absorbing a photon.
Remember that by assumption 3 we are dealing only
with particle character or identity, not momentum.

 $\bullet$  Suppose seventhly that for each particle $A$ there is an `anti-particle'
$A^{-1}$ for which $A \bigodot A^{-1} \rightarrow Q$.
We have defined a group structure, which we may take to be finite.

  Now I ask:  What continuous symmetries exist in this model?  This is
something of a reversal of the standard approach which examines
representations of a group to find particles.  I am looking at a finite
group representing preon interactions and searching for the observed
continuous symmetry.  Eigenvectors of such a transformation could be
candidates for identification with observed particles.

 The specific question I wish to address is this:  Can the observed
symmetries (SU(3), SU(2), etc.) be generated in a natural way from
linear transformations over finite groups, where the structure of
the finite group is maintained?

 The answer, as I will show, is:  Yes, some of them can be
generated--perhaps all, I don't know.  The price is high, however--the 
finite groups which represent particle interactions
must be non-abelian.  This is contrary to observation and intuition, but
in one case explored in detail the eigenvectors of the
transformation which have constant
eigenvalues do in fact commute, and only those eigenvectors whose
eigenvalues are functions of the parameters fail to commute.  This
suggests that only
those particles whose preon contents are subject to change are those
which fail to commute.  This remains to be proven or disproven in general,
however.

 A second, as yet unanswered, question is:  Can one predict which
symmetries can be generated?  There is a hint that by examining the
group characters one can determine the order of continuous symmetries,
but this hasn't been fully explored yet.

\section{ Foundations}
 The method is to consider the elements of a finite group as the basis
of a vector space, and then study those transformations of the basis
elements for which the new bases, considered as elements of a new
finite group
themselves, form a group isomorphic to the original.  Each of the new
bases is a linear combination (complex coefficients) of the original
basis elements (elements of the group).  


 Consider a finite group $G$ with elements $c_i$.  Number them in some
arbitrary order, from $0$ to $N-1$, where $0$ is the label for the
identity element and $N$ is the order of the group.  Display the
group operation by 
$c_i \bigodot c_j \rightarrow c_k$.  

 I work with a space in which the elements $P$ are defined as
$\sum_i p_i \thinspace c_i$, where $p_i$ is a complex number (it could be some other
field, but I haven't addressed that question), and $c_i$ are the elements
of the original finite group.  Multiplication by a complex number is
naturally defined by $b \thinspace P $ as $\sum_i ( b \thinspace p_i )
\thinspace c_i$, and the sum of two elements $P$ and $Q$ is
$\sum_i ( p_i + q_i ) \thinspace c_i$.  There is also a natural operation between
the elements of this space given in terms of the group operation by
$P \bigotimes Q = \sum_i \sum_j p_i q_j (c_i \bigodot c_j )$.

 
 In this space consider a linear transformation of the
original group elements, 
so that we have a new group $G^\prime$ with
$c_i^\prime \bigotimes c_j^\prime \rightarrow c_k^\prime$.  
This is an automorphism over the new space.  Let the
array $V$ be defined by
$c_i^\prime = V_{i,r} c_r$ so we can write
\begin{equation}
V_{i,r} c_r \bigotimes V_{j,s} c_s \rightarrow \phi_{k} V_{k,t} c_t
\end{equation}
\begin{equation}
V_{i,ts^{-1}} c_{ts^{-1}} \bigotimes V_{j,s} c_s \rightarrow V_{ij,t} c_t
\end{equation}
\begin{equation}
V_{i,ts^{-1}} c_{ts^{-1}} \bigotimes V_{j,s} c_s \rightarrow \phi_{ij} V_{ij,t} c_t
\end{equation}
\begin{equation}
V_{i,ts^{-1}} \thinspace V_{j,s} = \phi_{ij} V_{ij,t} = V_{i,r} \thinspace V_{j,r^{-1}t}
\end{equation}


Note that I assume that $V$ is complex.  One could restrict it to
be real, or use some other field entirely, but I'll take the
complex case.  It turns out that the $\phi_{ij}$ are
all identical, and can be divided out as a completely independent phase
(U(1), in other words).
%%%%You may verify that using the
%%%%equation ${V^{*}}_{i,r} V_{j,r^{-1}t} = V_{ij,t}$ changes nothing.
In what follows I will denote the
identity by 0, for simplicity and clarity.  The top element of a column
vector will correspond to this 0-element.  In what follows, if an
object has more than one subscript, these will be separated by commas.
If several terms are concatenated in a subscript without commas,
group addition of the specified elements is assumed.
  First let me demonstrate that the $\phi_x$ are identical.
Clearly $V$ must not be singular, or the
resulting transformed group would be smaller than the original.

 Define $f_i = \sum_{x} V_{i,x}$.  Then since
$$
\sum_{r} V_{i,r} V_{j,r^{-1}t} = V_{ij,t} \phi_{ij}
$$
$$
\sum_{r,t} V_{i,r} V_{j,r^{-1}t} = f_i f_j = \sum_{t} V_{ij,t} \phi_{ij} =
\phi_{ij} f_{ij}
$$
When $i = 0$, we get $f_0 \thinspace f_j = \phi_j \thinspace f_j$.
If any $f_k$ were 0, all would be, and thus V would be singular.
Since V is non-singular,
$f_j \neq 0$, so $f_0 = \phi_j \forall j$, thus all the $\phi_j$ are the
same, and will be called simply $\phi$, which is also clearly non-zero.

 To continue, restate the equation of $f$'s as $f_i \thinspace f_j =
f_{ij} \thinspace \phi$.  Then $f_{i^2} = f^2_i /\phi$.  This extends
to $f_{i^n} = f^n_i (1/\phi)^{n-1}$.  Now we know that for each $i$
an element in the group
there exists some $N$ such that $i^N = 0$ (the identity), since this
is a finite group.  Then $f_{i^N} = f_0 = \phi = f^N_i (1/\phi)^{N-1}$,
or, rearranging, $f^N_i = \phi^N$.  Thus
$$
f_i = \phi e^{2 i \pi {n_i \over N}} \quad\quad 0 \leq n_i \leq N
$$
But notice that the $n_i$ are discrete, and this is a continuous
transformation.  The identity is obviously a member of the set of 
transformations starting from the identity,
and for the identity each
$f_i$ is 1.  If we parameterize starting from the identity transform, then 
each $n_i$ must be $0$.  Therefore,
$$
\sum_x V_{i,x} = f_i = \phi \quad \forall i
$$

 Let $g_i = \sum_x V_{x,i}$.  Then
$$
\sum_{i,r} V_{i,r} \thinspace V_{j,r^{-1}t} = \sum_t V_{ij,t} \phi =
\sum_r g_r \thinspace V_{j,r^{-1}t} = g_t \phi
$$
$$
\sum_{s} V_{j,s} g_{s^{-1}} = g_0 \phi \quad \forall j
$$

 Let ${\bf G}_s = \{ g_{s^{-1}} \}$, a
column vector.  If ${\bf 1}$ is defined as a column vector of ones, then
we may write $V {\bf G} = g_0 \phi {\bf 1}$, which may be solved with
${\bf G} = g_0 \phi V^{-1} {\bf 1}$.  Note that $V^{-1}$ is also a
transformation, and the sum of elements in a row in it is equal to
some other $\phi^{\prime}$, which may be readily seen to be $\phi^*$.
Thus, ${\bf G} = g_0 \phi \phi^* {\bf 1}$, or ${\bf G} = g_0 {\bf 1}$.
Thus $g_i = g_0 \thinspace \forall i$.
$$
N g_0 = \sum_i g_i = \sum_{i,x} V_{x,i} = \sum_{x,i} V_{x,i} = \sum_x
f_x = N \phi \Rightarrow g_0 = \phi
$$

 For convenience, let us divide $V$ by $\phi$, so that we can quit
writing it out each time.  Let $V = V^{'} \thinspace \phi$.  Then
$$
V_{i,r} \thinspace V_{j,r^{-1}t} = V^{'}_{i,r} \thinspace V^{'}_{j,r^{-1}t}
\thinspace \phi^2 = V_{ij,t} \phi = V^{'}_{ij,t} \phi^2
$$
$$
V^{'}_{i,r} \thinspace V^{'}_{j,r^{-1}t} = V^{'}_{ij,t}
$$

 In what follows I will assume that the $\phi$ has been divided out, and
that the {\bf new} equations (dropping the primes) governing the transformation
are
\begin{equation}
V_{i,r} \thinspace V_{j,r^{-1}t} = V_{ij,t}
\quad \sum_i V_{i,j} = \sum_j V_{i,j} = 1
\end{equation}

\subsection{ Infinitesimal Transformations}
Now consider infinitesimal transformations away from the identity.  Here
$s = r^{-1}t$.
$$V_{i,ts^{-1}} V_{j,s} = V_{ij,t}$$
\begin{equation}
V_{a,b} \Rightarrow \delta_{a,b} + \delta V_{a,b}
\end{equation}
$$\left( \delta_{i,ts^{-1}} + \delta V_{i,ts^{-1}} \right) \left(
\delta_{j,s} + \delta V_{j,s} \right) = \delta_{ij,t} + \delta V_{ij,t}$$
$$\delta_{i,ts^{-1}} \delta_{j,s} + \delta_{j,s} \delta V_{i,ts^{-1}} +
\delta_{i,ts^{-1}} \delta V_{j,s} + O(\delta^2) =
\delta_{ij,t} + \delta V_{i,tj^{-1}} + \delta V_{j,i^{-1}t} =
\delta_{ij,t} + \delta V_{ij,t}$$
resulting in
\begin{equation}
\delta V_{i,tj^{-1}} + \delta V_{j,i^{-1}t} = \delta V_{ij,t} \quad , \quad
\sum_i \delta V_{i,j} = \sum_j \delta V_{i,j} = 0
\end{equation}
 The above are the fundamental equations governing the transformations of
the array.
 Let $i = j = 0$.  Then the fundamental equation reduces to
$2\delta V_{0,t} = \delta V_{0,t}$, so
\begin{equation}
\delta V_{0,t} = 0
\end{equation}
 Let $t = j$.  Then the fundamental equation becomes
\begin{equation}
\delta V_{i,0} + \delta V_{j,i^{-1}j} = \delta V_{ij,j}
\end{equation}
 Now for any $i \neq 0$ we know that there exists some $N \geq 2$ such that
$i^N = 0$, where $0$ is the identity.  Thus we make the following substitutions
\begin{eqnarray}
j = i \;\; \Rightarrow \;\; 2 \delta V_{i,0} = \delta V_{i^2,i} \\
j = i^2 \;\; \Rightarrow \;\; \delta V_{i,0} + \delta V_{i^2,i} =
\delta V_{i^3,i^2} = 3 \delta V_{i,0} \\
j = i^N \;\; \Rightarrow \;\; N \delta V_{i,0} = \delta V_{i^N,i^{N-1}}
= \delta V_{0,i^{N-1}} \\
\end{eqnarray}
 But since $\delta V_{0,x} = 0$, we must have
\begin{equation}
\delta V_{i,0} = 0 \;\; \forall i
\end{equation}
 The column vector in $\delta V$ corresponding to this is all zero's
($ \delta 1 = 0$ for the (0,0) position),
and the top row is also all zero's.
Remember that I am placing the identity element in the first
position.  We formally express $V$ as $ exp( \sum c_x \delta V_x)$
where the $\delta V_x$ are independent infinitesimal transformations away
from the initial value.  Therefore, since the initial value of $V$ is the identity, which
has zeros off the diagonal element in this row and column, any product of
any $\delta V$'s with the initial value of $V$ must continue to have zero's
in these positions.  Thus all $V$'s have $V_{0,0} = 1$, and $V_{x,0} = V_{0,x} = 0$.
\begin{equation}
V = \left( \begin{array}{cc} 1 & {0 \ldots 0} \\ 0 & \\ {\vdots} & V' \\
0 & \end{array} \right) \end{equation}

 Therefore, the identity element does not transform.


From the fundamental rule governing the continuous transformations,
$$
\delta V_{i,tj^{-1}} + \delta V_{j,i^{-1}t} = \delta V_{ij,t}
$$
if we let $t=0$ we have
$$
\delta V_{i,j^{-1}} + \delta V_{j,i^{-1}} = 0
$$
or
\begin{equation}
\delta V_{i,q} = -\delta V_{q^{-1},i^{-1}} \quad \forall i,q
\end{equation}

 Now restate the first fundamental equation, substituting $t = jq$.
$$\delta V_{i,jqj^{-1}} + \delta V_{j,i^{-1}jq} = \delta V_{ij,jq}$$
{\bf If $j$ commutes with all other elements of the group}, then
$jqj^{-1} = q$, and $ij = ji$, so
$$\delta V_{i,q} + \delta V_{j,ji^{-1}q} = \delta V_{ji,jq}$$
Now consider the sequence of equations generated by substituting for
$i = jI$ and $q = jQ$.
$$\delta V_{j^2I,j^2Q} = \delta V_{j,ji^{-1}q} + \delta V_{jI,jQ} =
\delta V_{j,ji^{-1}q} + \delta V_{I,Q} + \delta V_{j,ji^{-1}q} =
2 \delta V_{j,ji^{-1}q} + \delta V_{I,Q}$$
$$
\delta V_{j^3I,j^3Q} = 3 \delta V_{j,ji^{-1}q} + \delta V_{I,Q}
$$
and so on.  For some $N$, $j^N = 0$, and we get
$$
\delta V_{j^NI,j^NQ} = N \delta V_{j,ji^{-1}q} + \delta V_{I,Q} =\delta V_{I,Q}
$$
from which we get $\delta V_{j,ji^{-1}q} = 0$.  Since this is true for
arbitrary $i$ and $q$, we see that $\delta V_{j,x} = 0 \thinspace \forall x$.
Now since $j^{-1}$ will also commute, the same reasoning applied to it, and
we have $\delta V_{j^{-1},x} = 0 \thinspace \forall x$, which implies
$\delta V_{x^{-1},j} = 0 \thinspace \forall x$.  Thus {\bf if $j$ commutes
with all other elements of the group, $j$ is not transformed.}

 In an abelian group, any $j$ commutes with all other elements, so the above
is true for all $j$ in the group.  Thus
{\bf abelian groups are not continuously transformable}.  One 
may further state that
Abelian sub-groups do not ``internally" transform:  the matrix elements
of the $\delta V$ array which connect one element in an abelian sub-group
with another element in the subgroup are 0.
This is trivial for a subgroup of order 2.

 Clearly if this model does correspond to a physical system, it is not
an immediately intuitive one.

\subsection{ Further Simplifications}
 Now let us look at diagonal elements.  In the fundamental equation of
transformation, above, substitute $j=i$ and $t=i^2$.  We then get
$2\delta V_{i,i} = \delta V_{i^2,i^2}$.  If $j=i^2$, then we find
$\delta V_{i,i} + 2\delta V_{i^2,i^2} = \delta V_{i^3,i^3}$, and so
on.  This gives $N\delta V_{i,i} = \delta V_{i^N,i^N}$, but since
for some $N$, $i^N =0$, we must have
\begin{equation}
\delta V_{i,i} = 0 \quad \forall i
\end{equation}

 In another look at the fundamental equation, set $t=j$.  This results in
\begin{equation}
\delta V_{ij,j} = \delta V_{j,i^{-1}j}
\end{equation}

 Alternatively, if we set $ij \equiv q$, then we have 
\begin{equation}
\delta V_{i,tq^{-1}i} + \delta V_{i^{-1}q,i^{-1}t} = \delta V_{q,t}
\end{equation}
If $t = i$, then
\begin{equation}
\delta V_{i,iq^{-1}i} = \delta V_{q,i}
\end{equation}
Instead of reducing, we can use the inversion derived earlier ($\delta V_{j,r} =
-\delta V_{j^{-1},r^{-1}}$), and find that
$-\delta V_{i^{-1}qt^{-1},i^{-1}} + \delta V_{i^{-1}q,i^{-1}t} = \delta V_{q,t}$.
Since $i$ is arbitrary, 
substitute $j$ for $i^{-1}$ to find
\begin{equation}
\delta V_{jq,jt} - \delta V_{jqt^{-1},j} = \delta V_{q,t} \quad \forall j
\end{equation}
These equations are powerful tools, since they say that for any $q$ and
$t$ one has N-1 combinations of matrix elements all equal to the single
matrix element $\delta V_{q,t}$.

\subsection{ Conjugacy Classes and Eigenvectors}

 Each element $f$ in the group is a member of some conjugacy class,
which is the set $F$ of all elements in the group such that
if $r$ is in $F$, then there exists some $g$ in the group for which
$g \thinspace r \thinspace g^{-1} = f$.  One interesting question
(motivated by looking at a few examples) is `How does the sum of
members of a conjugacy class transform?'  The answer, as is shown
below, is that such a sum does remains the same under the differential
transformation used so far.

 The fundamental equation is
$$\delta V_{i,tj^{-1}} + \delta V_{j,i^{-1}t} = \delta V_{ij,t}$$
Substitute for $t$ the quantity $j^2 \thinspace s \thinspace j^{-1}$, and
sum the $s$ over all members of its particular conjugacy class, which
I'll call $S$.  The result is:
$$\sum_{s \in S} \delta V_{i,j^2 s j^{-2}} +
\sum_{s \in S} \delta V_{j,i^{-1}j^2 s j^{-1}} =
\sum_{s \in S} \delta V_{ij,j^2 s j^{-1}}$$
Since we are summing over all members of a conjugacy class, we can
replace $j^2 s j^{-2}$ in the first term with $s$, $i^{-1}j^2 s j^{-1}$
in the second term with $i^{-1}j s$, and $j^2 s j^{-1}$ on the
right side with $j s$.  This results in the much simpler
$$\sum_{s \in S} \delta V_{i,s} +
\sum_{s \in S} \delta V_{j,i^{-1}j s} =
\sum_{s \in S} \delta V_{ij,j s}$$
Now if we set $j=i$, we find
$$2\sum_{s \in S} \delta V_{i,s} =
\sum_{s \in S} \delta V_{i^2,i s}$$
For $j=i^2$, we get
$$\sum_{s \in S} \delta V_{i,s} +
\sum_{s \in S} \delta V_{i^2,i s} =
\sum_{s \in S} \delta V_{i^3,i^2 s} = 3\sum_{s \in S} \delta V_{i,s}$$
If $n >0$, we have by induction for $j=i^n$
$$\sum_{s \in S} \delta V_{i^{n+1},i^n s} = (n+1)\sum_{s \in S} \delta V_{i,s}$$
For some $m$, $i^m = 0$, so for $n=m-1$
$$\sum_{s \in S} \delta V_{i^m,i^{m-1} s} = m\sum_{s \in S} \delta V_{i,s}$$
But since $\delta V_{0,x} = 0$, the left side is 0, and thus
$$\sum_{s \in S} \delta V_{i,s} = 0$$

 These sums of elements in a conjugacy class are thus eigenvectors of the
transformation $V$, with constant eigenvalue 1.  Such sums of elements
not only do not transform under $V$, but they commute with any other
linear combination of elements of the group.

\subsection{ Discrete Transforms: Permutations}

  We can have transformations which are simply permutations of the
group elements, as well as permutations with a sign.  So long as these
preserve the group structure they are legitimate objects of study here.
Some of these permutations may arise naturally from continuous transformations
from the identity, but some do not.  Those which do not arise from
transformations from the form
a finite group themselves, and each member of this set of
group-preserving permutations may serve as
the basis for a family of continuous transformations over the original
group.

 Clearly elements of a conjugacy class must either map
into each other or into elements of a conjugacy class of the same size.
This helps restrict the number of cases to examine.


 Let's returning to our physical
model--preons with no well-defined quantum numbers which are combined to
form objects which DO have well-defined quantum numbers.  Presumably
the constant eigenvectors of the transformation correspond to real particles, or
things from which real particles could be generated.  Since most of
the constant eigenvectors are sums of elements (`preons') from the
same `conjugacy class', it isn't clear 
that we have a natural way to find anti-particles
from preons without the use of `signed permutation transforms'.

 In any case the model is becoming somewhat
unwieldy, with groups up to order 16 studied without finding the
SU(3) color group.  A preon model with more preons than particles is
unaesthetic, not to mention dubious.


%%%%%% Example section:  SU(2) or SO(3) or allied group.  SQUARE/TRIANGLE
\section{ Simple Examples}
%%%% TRIANGLE %%%%%%%%%%%%%%%%%%%%%%%%%%%%%%%%%%%%%%%%%%

 {\bf I am a firm believer in the power of examples, and will generate
a number of them for use in checking hypotheses.}

 {\bf Consider the group of the symmetries of an equilateral triangle. 
It has 6 elements.}  $O_3$ 

 We may designate the elements of the group by \G0 , \G1 , \G2 , \G3 , \G4 ,
and \G5 , where \G0 is the identity,
$a \equiv \G4$, $b \equiv \G1$, $b^2 \equiv \G2$,
and so on.  We may define a product (Cayley) table for this group in the following
way:
\begin{equation}
\begin{tabular}{c|cccccccc}
$\bigodot$ & \G0 & \G1 & \G2 & \G3 & \G4 & \G5 \\ \hline
\G0 & \G0 & \G1 & \G2 & \G3 & \G4 & \G5 \\
\G1 & \G1 & \G2 & \G0 & \G4 & \G5 & \G3 \\
\G2 & \G2 & \G0 & \G1 & \G5 & \G3 & \G4 \\
\G3 & \G3 & \G5 & \G4 & \G0 & \G2 & \G1 \\
\G4 & \G4 & \G3 & \G5 & \G1 & \G0 & \G2 \\
\G5 & \G5 & \G4 & \G3 & \G2 & \G1 & \G0
\end{tabular}
\end{equation}
I will dispense with the labels above and to the left,
since the group operation with \G0 easily identifies which
element is which.

 It may be shown, via fairly tedious and trivial algebra, that all
elements of $\delta V$ are zero except a few which are all simply related
to each other.  If there are some tiny changes $\alpha$, $\beta$, and $\gamma$;
then
\begin{equation}
\delta V = \left(
\begin{array}{cccccc}
 0 &  0 &  0 &  0 &  0 &  0 \\
 0 &  0 &  0 &  \alpha &  -\beta &  -\alpha+\beta \\
 0 &  0 &  0 &  -\alpha &  \beta &  \alpha-\beta \\
 0 &  \alpha &  -\alpha &  0 &  \gamma &  -\gamma \\
 0 &  -\beta &  \beta &  -\gamma &  0 &  \gamma \\
 0 &  -\alpha+\beta &  \alpha-\beta & \gamma &  -\gamma &  0 \\
\end{array}
\right) 
\end{equation}
\begin{equation}
\delta V
\equiv \left( \begin{array}{cc} 0 & A \\ A^T & 0 \end{array} \right) \alpha +
\left( \begin{array}{cc} 0 & B \\ B^T & 0 \end{array} \right) \beta +
\left( \begin{array}{cc} 0 & 0 \\ 0 & C \end{array} \right) \gamma
\end{equation}
where
$$
a \equiv 
\left( \begin{array}{ccc} 0 & 0 & 0 \\ 1 & 0  & -1 \\ -1 & 0 & 1 \\ \end{array} \right) 
\quad \quad , \quad \quad
b \equiv 
\left( \begin{array}{ccc} 0 & 0 & 0 \\ 0 & -1 & 1 \\ 0 & 1 & -1 \\ \end{array} \right) \quad \quad
b^T = b
$$
$$
c \equiv 
\left( \begin{array}{ccc} 0 & 1 & -1 \\ -1 & 0 & 1 \\ 1 & -1 & 0 \\ \end{array} \right) \quad \quad
c^T = -c
$$
\begin{equation}
A = \left( \begin{array}{cc} 0 & a \\ a^T & 0 \\ \end{array} \right) \quad
B = \left( \begin{array}{cc} 0 & b \\ b^T & 0 \\ \end{array} \right) \quad
C = \left( \begin{array}{cc} 0 & 0 \\ 0 & c \\ \end{array} \right)
\end{equation}
 The eigenvectors corresponding to eigenvalue 0 of $\delta V$ are
\begin{displaymath}
\left( \begin{array}{c} 1 \\ 0 \\ 0 \\ 0 \\ 0 \\ 0 \end{array}\right)
\left( \begin{array}{c} 0 \\ 1 \\ 1 \\ 0 \\ 0 \\ 0 \end{array}\right)
\left( \begin{array}{c} 0 \\ 0 \\ 0 \\ 1 \\ 1 \\ 1 \end{array}\right)
\left( \begin{array}{c} 0 \\ \gamma \\ -\gamma \\ \alpha-\beta \\ \beta-\alpha \\ \alpha+\beta \end{array}\right)
\end{displaymath}
 Notice that these eigenvectors commute under the defined field, as expected,
as they are sums of the elements in the conjugacy classes.
$$
\left[ A,B \right] = -2C \quad \quad
\left[ B,C \right] = 2A + B \quad \quad
\left[ A,C \right] = -A - 2B
$$
If we set
$$
X \equiv {i \over 2} (A+B) \quad \quad
Y \equiv {1 \over \sqrt{3}} C \quad \quad
Z \equiv {i \over 2 \sqrt{3}} (A-B)
$$
Then we find
$$
\left[ X,Y \right] = Z \quad\quad
\left[ Y,Z \right] = X \quad\quad
\left[ X,Z \right] = Y \quad\quad
$$
Let us now define
$$
e \equiv A \cos \theta + B + C \sin \theta \quad \quad
f \equiv A \cos \theta - B + C \sin \theta \quad \quad
h \equiv A 2 \sin \theta - B 2 \cos \theta
$$
The resulting commutation relations are
$$
\left[ e,f \right] = h \quad\quad
\left[ h,e \right] = 2 e \quad\quad
\left[ h,f \right] = -2f
$$
which are the conditions for generating the simplest Kac-Moody
algebra \footnote{\underline{Infinite Dimensional Lie Algebras},
Victor G. Kac, page x; suggested by Georgia Benkart}.
If we set
$$
T_1 \equiv {1 \over 2} (A+B) \quad \quad
T_2 \equiv {1 \over 2 \sqrt{3}} (A-B) \quad \quad
T_3 \equiv {-i \over \sqrt{3}} C
$$
then we have $\left[ T_i,T_j \right] = i \epsilon_{ijk} T_k$
this is related to the generators for SU(2) or SO(3).

 Look at powers of $\delta V$.  The second power is 
\begin{equation}
{\delta V}^2 = \left(
\begin{array}{cccc}
 0 &  0 &  0 &  0 \\
 0 &  2\left(\alpha^2+\beta^2-\alpha\beta\right) &  -2\left(\alpha^2+\beta^2-\alpha\beta\right)  &  
\gamma \left(2\beta-\alpha\right) \\
 0 &  -2\left(\alpha^2+\beta^2-\alpha\beta\right)  &  2\left(\alpha^2+\beta^2-\alpha\beta\right)  &
\gamma \left(\alpha-2\beta\right) \\
 0 &  \gamma \left(\alpha-2\beta\right) &  \gamma \left(2\beta-\alpha\right) &
2\left(\alpha^2-\gamma^2\right) \\
 0 &  -\gamma\left(2\alpha-\beta\right) &  \gamma\left(2\alpha-\beta\right) &
-2\alpha\beta+\gamma^2 \\
0 &  \gamma\left(\alpha+\beta\right)  &  -\gamma\left(\alpha+\beta\right) &
2\left(-\alpha^2+\alpha\beta\right)+\gamma^2 \\
\end{array} \right.
\end{equation}
$$
\left.
\begin{array}{cc}
 0 &  0 \\
\gamma\left(2\alpha-\beta\right) &  -\gamma\left(\alpha+\beta\right) \\
-\gamma\left(2\alpha-\beta\right) & \gamma\left(\alpha+\beta\right) \\
-2\alpha\beta+\gamma^2 &  -2\left(\alpha^2-\alpha\beta\right)+\gamma^2 \\
2\left(\beta^2-\gamma^2\right) & -2\left(\beta^2-\alpha\beta\right)+\gamma^2 \\
-2\left(\beta^2-\alpha\beta\right)+\gamma^2 & 
2\left(\alpha^2-2\alpha\beta+\beta^2\right)-2\gamma^2 \\
\end{array}
\right) 
$$
 To find $V$, we use $V = e^{\delta V}$.

% Hmm.  double check this stuff . . . .
% The third power is
%\begin{equation}
%{\delta V}^3 = 4\left(\alpha^2+\beta^2-\gamma^2\right)
%{\delta V}
%\end{equation}
%We can easily solve the series.  It is well-defined and converges,
%but clearly the value can become arbitrarily large for arbitrarily
%large values of $\alpha$ or $\beta$.  

 The eigenvalues of the array $\delta V$ are given by
\begin{equation}
0 = X^4 \left( 4 \alpha^2 - 4\alpha\beta + 4\beta^2 - 3\gamma^2 - X^2 \right)
\end{equation}

 The eigenvalues of $\exp (\delta V)$ thus have 4 $1'$s, and 2 others which
may be real or imaginary depending on $\alpha$, $\beta$, and $\gamma$.

  There can be transformations which consist of simple permutations of the
elements, or of permutations with a sign.  For this group there are six
unsigned permutations, which may or may not be special cases of the above
continuous transform.  For those which are {\bf not} special cases, the
continuous transform may be simply applied to these permutations to get
new families of transformations.

  The first set of permutations is given by the mapping \G0 $\rightarrow$ \G0,
\G1 $\rightarrow$ \G1, \G2 $\rightarrow$ \G2, and the following 3 cases:  
The identity
(\G3 $\rightarrow$ \G3, \G4 $\rightarrow$ \G4, and \G5 $\rightarrow$ \G5), 
a left rotation
(\G3 $\rightarrow$ \G4, \G4 $\rightarrow$ \G5, and \G5 $\rightarrow$ \G3), 
and a right rotation
(\G3 $\rightarrow$ \G5, \G4 $\rightarrow$ \G3, and \G5 $\rightarrow$ \G4).  These
can be derived from transformations from the identity.

%% The second set of permutations, which are clearly not transformations
%% from the identity, is given by mapping 
Another set of permutations is also reachable from transformations from the identity:
\G0 $\rightarrow$ \G0, \G1 $\rightarrow$ \G2, and \G2 $\rightarrow$ \G1, together
with the following 3 cases: (\G3 $\rightarrow$ \G3, \G4 $\rightarrow$ \G5,
and \G5 $\rightarrow$ \G4), (\G3 $\rightarrow$ \G4, \G4 $\rightarrow$ \G3,
and \G5 $\rightarrow$ \G5), and (\G3 $\rightarrow$ \G5, \G4 $\rightarrow$ \G4,
and \G5 $\rightarrow$ \G3).

%%   In addition there are permutations possible with a sign.  We can
%% consistently map elements \G3, \G4, and \G5 to -\G3, -\G4, and -\G5
%% respectively, giving a total of 12 basis permutations.

 {\bf Consider the group of the symmetries of a square.}  It has 8 elements, and
may be defined by the following equations:
\begin{equation}
a^2 \rightarrow 0 \quad , \quad b^4 \rightarrow 0 \quad , \quad b a \rightarrow
a b^3
\end{equation}

 We may designate the elements of the group by \G0 , \G1 , \G2 , \G3 , \G4 ,
\G5 , \G6 , and \G7 , where \G0 is the identity,
$a \equiv \G4$, $b \equiv \G1$, $b^2 \equiv \G2$,
and so on.  We may define a product (Cayley) table for this group in the following
way:
\begin{equation}
\begin{tabular}{c|cccccccc}
$\bigodot$ & \G0 & \G1 & \G2 & \G3 & \G4 & \G5 & \G6 & \G7 \\ \hline
\G0 & \G0 & \G1 & \G2 & \G3 & \G4 & \G5 & \G6 & \G7 \\
\G1 & \G1 & \G2 & \G3 & \G0 & \G5 & \G6 & \G7 & \G4 \\
\G2 & \G2 & \G3 & \G0 & \G1 & \G6 & \G7 & \G4 & \G5 \\
\G3 & \G3 & \G0 & \G1 & \G2 & \G7 & \G4 & \G5 & \G6 \\
\G4 & \G4 & \G7 & \G6 & \G5 & \G0 & \G3 & \G2 & \G1 \\
\G5 & \G5 & \G4 & \G7 & \G6 & \G1 & \G0 & \G3 & \G2 \\
\G6 & \G6 & \G5 & \G4 & \G7 & \G2 & \G1 & \G0 & \G3 \\
\G7 & \G7 & \G6 & \G5 & \G4 & \G3 & \G2 & \G1 & \G0
\end{tabular}
\end{equation}
In the future I will dispense with the labels above and to the left,
since the group operation with \G0 easily identifies which
element is which.

 It may be shown, via fairly tedious and trivial algebra, that all
elements of $\delta V$ are zero except a few which are all simply related
to each other.  Let there be some tiny changes $\alpha$, $\beta$, and $\gamma$;
then
\begin{equation}
\delta V = \left(
\begin{array}{cccccccc}
 0 &  0 &  0 &  0 &  0 &  0 &  0 &  0 \\
 0 &  0 &  0 &  0 &  \alpha &  \beta &  -\alpha &  -\beta \\
 0 &  0 &  0 &  0 &  0 &  0 &  0 &  0 \\
 0 &  0 &  0 &  0 &  -\alpha &  -\beta &  \alpha &  \beta \\
 0 &  \alpha &  0 &  -\alpha &  0 &  \gamma &  0 &  -\gamma \\
 0 &  \beta &  0 &  -\beta &  -\gamma &  0 &  \gamma &  0 \\
 0 &  -\alpha &  0 &  \alpha &  0 &  -\gamma &  0 &  \gamma \\
 0 &  -\beta &  0 &  \beta &  \gamma &  0 &  -\gamma &  0
\end{array}
\right) 
\end{equation}
\begin{equation}
\delta V 
\equiv \left( \begin{array}{cc} 0 & A \\ A^T & 0 \end{array} \right) 2\alpha +
\left( \begin{array}{cc} 0 & B \\ B^T & 0 \end{array} \right) 2\beta +
\left( \begin{array}{cc} 0 & 0 \\ 0 & C \end{array} \right) 2\gamma
\end{equation}
where
$$
A \equiv {1 \over 2} 
\left( \begin{array}{cc} a & -a \\ -a & a \\ \end{array} \right) \quad \quad
a \equiv \left( \begin{array}{cc} 1 & 0 \\ 0 & 0 \\ \end{array} \right)
$$
$$
B \equiv {1 \over 2} 
\left( \begin{array}{cc} b & -b \\ -b & b \\ \end{array} \right) \quad \quad
b \equiv \left( \begin{array}{cc} 0 & 0 \\ 0 & 1 \\ \end{array} \right) \quad \quad
B^T = B
$$
$$
C \equiv {1 \over 2} 
\left( \begin{array}{cc} c & -c \\ -c & c \\ \end{array} \right) \quad \quad
c \equiv \left( \begin{array}{cc} 0 & 1 \\ -1 & 0 \\ \end{array} \right) \quad \quad
C^T = -C
$$
\begin{equation}
\delta V^3 = 4 \left( \alpha^2 + \beta^2 - \gamma^2 \right) \delta V
\end{equation}
 Notice that
$$
\left[ A,B \right] = C \quad \quad
\left[ C,B \right] = A \quad \quad
\left[ A,C \right] = B
$$
Let $X=A$, $Y=C$, and $Z=B$.  Then
$$
\left[ X,Y \right] = Z \quad \quad
\left[ Y,Z \right] = X \quad \quad
\left[ X,Z \right] = Y
$$
and this has the same structure as the `triangle group'.

 To find $V$, we use $V = e^{\delta V}$.

% The third power is
%\begin{equation}
%{\delta V}^3 = \left(4\alpha^2+4\beta^2-4\alpha\beta-3\gamma^2\right)
%{\delta V}
%\end{equation}
%We can easily solve the series.  It is well-defined and converges,
%but clearly the value can become arbitrarily large for arbitrarily
%large values of $\alpha$ or $\beta$.  

 The eigenvalues of the $\delta V$ array are given by
\begin{equation}
0 = X^6 \left( X^2 - 4\left(\alpha^2+\beta^2-\gamma^2 \right) \right)
\end{equation}

 The eigenvalues of $\exp(\delta V)$ have then 6 $1$'s and two others
which are either real or imaginary, depending on the values of
$\alpha$, $\beta$, and $\gamma$.

%%%% THE QUATERNION GROUP %%%%%%%%%%%%%%%%%
\section{ An Example with the Quaternion Group}
 Consider the quaternion group.  It has 8 elements, and
may be defined by the following equations:
\begin{equation}
a^4 \rightarrow 0 \quad , \quad b^2 \rightarrow a^2 \quad , \quad b a \rightarrow
a^3 b
\end{equation}

 We may designate the elements of the group by \G0 , \G1 , \G2 , \G3 , \G4 ,
\G5 , \G6 , and \G7 , where \G0 is the identity,
$a^2 \equiv \G1$, $a \equiv \G2$, $a^3 \equiv \G3$,
$b \equiv \G4$, $a^2 b \equiv \G5$, $a b \equiv \G6$, and $a^3 b \equiv
\G7$.  We may define a product (Cayley) table for this group in the following
way:
\begin{equation}
\begin{array}{cccccccc}
\G0 & \G1 & \G2 & \G3 & \G4 & \G5 & \G6 & \G7 \\
\G1 & \G0 & \G3 & \G2 & \G5 & \G4 & \G7 & \G6 \\
\G2 & \G3 & \G1 & \G0 & \G6 & \G7 & \G5 & \G4 \\
\G3 & \G2 & \G0 & \G1 & \G7 & \G6 & \G4 & \G5 \\
\G4 & \G5 & \G7 & \G6 & \G1 & \G0 & \G2 & \G3 \\
\G5 & \G4 & \G6 & \G7 & \G0 & \G1 & \G3 & \G2 \\
\G6 & \G7 & \G4 & \G5 & \G3 & \G2 & \G1 & \G0 \\
\G7 & \G6 & \G5 & \G4 & \G2 & \G3 & \G0 & \G1
\end{array}
\end{equation}

 Grinding through the algebra will show that all
elements of $\delta V$ are zero except a few which are described by
three independent tiny changes $\alpha$, $\beta$, and $\gamma$.
\begin{equation}
\delta V = \alpha \left(
\begin{array}{cccccccc}
 0 &  0 &  0 &  0 &  0 &  0 &  0 &  0 \\
 0 &  0 &  0 &  0 &  0 &  0 &  0 &  0 \\
 0 &  0 &  0 &  0 &  \alpha &  -\alpha &  \beta &  -\beta \\
 0 &  0 &  0 &  0 &  -\alpha &  \alpha &  -\beta &  \beta \\
 0 &  0 &  -\alpha &  \alpha & 0 & 0 &  \gamma &  -\gamma \\
 0 &  0 &  \alpha & -\alpha & 0 &  0 &  -\gamma &  \gamma \\
 0 &  0 &  -\beta &  \beta &  -\gamma &  \gamma &  0 &  0 \\
 0 &  0 &  \beta &  -\beta &  \gamma &  -\gamma &  0 &  0
\end{array}
\right)
\end{equation}
$$ {\rm Let}\quad a \equiv {1 \over 2}
\left( \begin{array}{cc} 1 & -1 \\ -1 & 1 \end{array} \right) \quad
a^2 = a$$
$$ A \equiv \left( \begin{array}{cccc} 0 & 0 & 0 & 0 \\ 0 & 0 & a & 0 \\
0 & -a & 0 & 0 \\ 0 & 0 & 0 & 0 \end{array} \right) \quad
B \equiv \left( \begin{array}{cccc} 0 & 0 & 0 & 0 \\ 0 & 0 & 0 & a \\
0 & 0 & 0 & 0 \\ 0 & -a & 0 & 0 \end{array} \right) \quad
C \equiv \left( \begin{array}{cccc} 0 & 0 & 0 & 0 \\ 0 & 0 & 0 & 0 \\
0 & 0 & 0 & a \\ 0 & 0 & -a & 0 \end{array} \right)$$
$$BA = \left( \begin{array}{cccc} 0 & 0 & 0 & 0 \\ 0 & 0 & 0 & 0 \\
0 & 0 & 0 & 0 \\ 0 & 0 & -a^2 & 0 \end{array} \right) \quad
AB = \left( \begin{array}{cccc} 0 & 0 & 0 & 0 \\ 0 & 0 & 0 & 0 \\
0 & 0 & 0 & -a^2 \\ 0 & 0 & 0 & 0 \end{array} \right) $$
$$AB-BA = \left( \begin{array}{cccc} 0 & 0 & 0 & 0 \\ 0 & 0 & 0 & 0 \\
0 & 0 & 0 & -a^2 \\ 0 & 0 & a^2 & 0 \end{array} \right) = -C$$
$$AC-CA = \left( \begin{array}{cccc} 0 & 0 & 0 & 0 \\ 0 & 0 & 0 & a^2 \\
0 & 0 & 0 & 0 \\ 0 & -a^2 & 0 & 0 \end{array} \right) = B$$
$$BC-CB = \left( \begin{array}{cccc} 0 & 0 & 0 & 0 \\ 0 & 0 & -a^2 & 0 \\
0 & a^2 & 0 & 0 \\ 0 & 0 & 0 & 0 \end{array} \right) = - A$$

 From the above, if we let
$$T_1 \equiv B \quad 
T_2 \equiv A \quad 
T_3 \equiv C$$
we see that
$$\left[ T_i, T_j \right] = \epsilon_{ijk} T_k$$
which describes the generators for the groups O(3) and SU(2).  At this
point it is not obvious which is generated here.  I cannot use real
coefficients to get this into Kac-Moody form, though I can do it with
complex ones ($h = 2 i B$, $e=iA+C$, and $f=iA-C$).

 To generate a transformation one may use the following method:
$$V = e^{\alpha A + \beta B + \gamma C}$$
$$\alpha A + \beta B + \gamma C = \left( \begin{array}{cccc}
0 & 0 & 0 & 0 \\ 0 & 0 & \alpha & \beta \\ 0 & -\alpha & 0 & \gamma \\
0 & -\beta & -\gamma & 0 \end{array} \right) \equiv Q$$
$$Q^2 = \left( \begin{array}{cccc} 0 & 0 & 0 & 0 \\
0 & -(\alpha^2 + \beta^2) & -\beta \gamma & \alpha \gamma \\
0 & -\beta \gamma & -(\alpha^2 + \gamma^2) & -\alpha \beta \\
0 & \alpha \gamma & -\alpha \beta & -(\beta^2 + \gamma^2)  \end{array}
\right) \quad \quad Q^3 = -Q \left( \alpha^2 + \beta^2 + \gamma^2 \right)$$
 In the above, every non-zero matrix element is understood to be
multiplied by $a$, where $a$ is the $2 \times 2$ array defined above.
\begin{equation}
V(\alpha, \beta, \gamma) = I + {Q \over r} \sin{r} +
{Q^2 \over r^2} \left( 1 - \cos{r} \right) \quad {\rm where} \quad
r \equiv \sqrt{\alpha^2 + \beta^2 + \gamma^2}
\end{equation}
Define $\epsilon \equiv (1-\cos{r})$, and $\delta \equiv \sin{r}$.
In its full glory, the array expands to the following:
$$
\left( \begin{array}{cccccccc}
1 & 0 & 0 & 0 & 0 & 0 & 0 & 0 \\
0 & 1 & 0 & 0 & 0 & 0 & 0 & 0 \\
0 & 0 & 1 - {\alpha^2+\beta^2 \over 2 r^2} \epsilon &
{\alpha^2 + \beta^2 \over 2 r^2} \epsilon &
{\alpha \over 2 r}\delta - {\beta \gamma \over 2 r^2}\epsilon &
-{\alpha \over 2 r}\delta + {\beta \gamma \over 2 r^2}\epsilon &
{\beta \over 2 r}\delta + {\alpha \gamma \over 2 r^2}\epsilon &
-{\beta \over 2 r}\delta -{\alpha \gamma \over 2 r^2}\epsilon \\
0 & 0 & {\alpha^2 + \beta^2 \over 2 r^2} \epsilon &
1 - {\alpha^2+\beta^2 \over 2 r^2} \epsilon &
-{\alpha \over 2 r}\delta + {\beta \gamma \over 2 r^2}\epsilon &
{\alpha \over 2 r}\delta - {\beta \gamma \over 2 r^2}\epsilon &
-{\beta \over 2 r}\delta -{\alpha \gamma \over 2 r^2}\epsilon &
{\beta \over 2 r}\delta + {\alpha \gamma \over 2 r^2}\epsilon \\
0 & 0 &
-{\alpha \over 2 r}\delta - {\beta \gamma \over 2 r^2}\epsilon &
{\alpha \over 2 r}\delta + {\beta \gamma \over 2 r^2}\epsilon &
1 - {\alpha^2 + \gamma^2 \over 2 r^2} \epsilon &
{\alpha^2 + \gamma^2 \over 2 r^2} \epsilon &
{\gamma \over 2 r}\delta - {\alpha \beta \over 2 r^2}\epsilon &
-{\gamma \over 2 r}\delta + {\alpha \beta \over 2 r^2}\epsilon \\
0 & 0 &
{\alpha \over 2 r}\delta + {\beta \gamma \over 2 r^2}\epsilon &
-{\alpha \over 2 r}\delta - {\beta \gamma \over 2 r^2}\epsilon &
{\alpha^2 + \gamma^2 \over 2 r^2} \epsilon &
1 - {\alpha^2 + \gamma^2 \over 2 r^2} \epsilon &
-{\gamma \over 2 r}\delta + {\alpha \beta \over 2 r^2}\epsilon &
{\gamma \over 2 r}\delta - {\alpha \beta \over 2 r^2}\epsilon \\
0 & 0 &
-{\beta \over 2 r}\delta + {\alpha \gamma \over 2 r^2}\epsilon &
{\beta \over 2 r}\delta - {\alpha \gamma \over 2 r^2}\epsilon &
-{\gamma \over 2 r}\delta - {\alpha \beta \over 2 r^2}\epsilon &
{\gamma \over 2 r}\delta + {\alpha \beta \over 2 r^2}\epsilon &
1 - {\beta^2 + \gamma^2 \over 2 r^2}\epsilon &
{\beta^2 + \gamma^2 \over 2 r^2}\epsilon \\
0 & 0 &
{\beta \over 2 r}\delta - {\alpha \gamma \over 2 r^2}\epsilon &
-{\beta \over 2 r}\delta + {\alpha \gamma \over 2 r^2}\epsilon &
{\gamma \over 2 r}\delta + {\alpha \beta \over 2 r^2}\epsilon &
-{\gamma \over 2 r}\delta - {\alpha \beta \over 2 r^2}\epsilon &
{\beta^2 + \gamma^2 \over 2 r^2}\epsilon &
1 - {\beta^2 + \gamma^2 \over 2 r^2}\epsilon
\end{array} \right)
$$
There are 5 eigenvalues of value 1, with constant eigenvectors.
They are all sums of the members of the 5 distinct conjugacy classes.
There is one additional eigenvalue of value 1, with variable
eigenvector.  The other two eigenvalues are $\lambda = e^{ir}$.
Notice that the eigenvalues are functions only of $r$, so in order
for the array $V$ to be a function of $\alpha$, $\beta$, and $\gamma$,
the eigenvectors must be functions of $\alpha$, $\beta$, and $\gamma$:
in other words, not constant.
\begin{equation}
\begin{array}{c} 1 \\ \left( 
\begin{array}{c} 1 \\ 0 \\ 0 \\ 0 \\ 0 \\ 0 \\ 0 \\ 0 \end{array} \right) \end{array}
\begin{array}{c} 1 \\ \left( 
\begin{array}{c} 0 \\ 1 \\ 0 \\ 0 \\ 0 \\ 0 \\ 0 \\ 0 \end{array} \right) \end{array}
\begin{array}{c} 1 \\ \left( 
\begin{array}{c} 0 \\ 0 \\ 1 \\ 1 \\ 0 \\ 0 \\ 0 \\ 0 \end{array} \right) \end{array}
\begin{array}{c} 1 \\ \left( 
\begin{array}{c} 0 \\ 0 \\ 0 \\ 0 \\ 1 \\ 1 \\ 0 \\ 0 \end{array} \right) \end{array}
\begin{array}{c} 1 \\ \left( 
\begin{array}{c} 0 \\ 0 \\ 0 \\ 0 \\ 0 \\ 0 \\ 1 \\ 1 \end{array} \right) \end{array}
\begin{array}{c} 1 \\ \left( 
\begin{array}{c} 0 \\ 0 \\ \gamma \\ -\gamma \\
-\beta \\ \beta \\ \alpha \\ -\alpha \end{array} \right) \end{array}
\end{equation}
\begin{equation}
\begin{array}{c} e^{ir} \\ \left( 
\begin{array}{c} 0 \\ 0 \\ \alpha^2 + \beta^2 \\ -(\alpha^2 + \beta^2) \\
(\beta \gamma + i \alpha r) \\ -(\beta \gamma + i \alpha r) \\
(-\alpha \gamma + i \beta r) \\ -(-\alpha \gamma + i \beta r) \end{array} 
\right) \end{array}
\begin{array}{c} e^{-ir} \\ \left( 
\begin{array}{c} 0 \\ 0 \\ \alpha^2 + \beta^2 \\ -(\alpha^2 + \beta^2) \\
(\beta \gamma - i \alpha r) \\ -(\beta \gamma - i \alpha r) \\ 
(-\alpha \gamma + i \beta r) \\ -(-\alpha \gamma - i \beta r) \end{array} 
\right) \end{array}
\end{equation}
Denote the eigenvectors in the order above by $e_i$.  The first 5 are
sums of members of the 5 distinct conjugacy classes, and commute with
all the rest.
Further, the product of any of $e_2$, $e_3$ or $e_4$ with any of $e_5$,
$e_6$, or $e_7$ is zero.  In addition, $e_1$ operating on any of the
last three gives the negative of that eigenvector.  The square of any of the
last three is a factor times $e_1 - e_0$.  The product of $e_6$ and $e_7$ is
the complex conjugate of the product of $e_7$ and $e_6$.  This
leaves only the products of $e_5$ with $e_6$ and $e_7$, which do not
commute.  


 Let us look further at the transformation, but restrict ourselves to
$\beta = \gamma = 0$.  Then $\alpha = r$, $\epsilon = 2\sin^2{\alpha/2}$,
and $\delta = 2 \sin{\alpha/2} \cos{\alpha/2}$.  Let $\mu \equiv
\cos^2{\alpha/2}$, $\nu \equiv \sin^2{\alpha/2}$, and $\omega \equiv
\sin{\alpha/2} \cos{\alpha/2}$.  Then the array reduces to
$$
\left( \begin{array}{cccccccc}
1 & 0 & 0 & 0 & 0 & 0 & 0 & 0 \\
0 & 1 & 0 & 0 & 0 & 0 & 0 & 0 \\
0 & 0 & \mu & \nu & \omega & -\omega & 0 & 0 \\
0 & 0 & \nu & \mu & -\omega & \omega & 0 & 0 \\
0 & 0 & -\omega & \omega & \mu & \nu & 0 & 0 \\
0 & 0 & \omega & -\omega & \nu & \mu & 0 & 0 \\
0 & 0 & 0 & 0 & 0 & 0 & 1 & 0 \\
0 & 0 & 0 & 0 & 0 & 0 & 0 & 1
\end{array} \right)
$$
When $\mu = \cos^2{\alpha/2} = 1$, then this becomes the identity.
When $\nu = 1$, the inner array becomes
$$
\Lambda = \left( \begin{array}{cccc}
0 & 1 & 0 & 0 \\
1 & 0 & 0 & 0 \\
0 & 0 & 0 & 1 \\
0 & 0 & 1 & 0
\end{array} \right)
$$
where $\Lambda$ swaps the elements $c_2$ and $c_3$ and the elements
$c_4$ and $c_5$.  This corresponds to eigenvalues of -1, and $r = \pi$.
$\Lambda^2 = I$, of course.  This suggests that
the group displayed is SO(3).

  There are permutations over the group elements possible here also.
For unsigned permutations, there are 3 cases which turn out to be
special cases of the above continous transform:  (\G2 $\rightarrow$ \G3
and \G4 $\rightarrow$ \G5), (\G2 $\rightarrow$ \G3 and \G6 $\rightarrow$ \G7),
and (\G4 $\rightarrow$ \G5 and \G6 $\rightarrow$ \G7).  The first of these
is illustrated directly above.  There are two additional cases which are
{\bf not} cases of the original transformation from the identity--and thus
may serve as bases for families of transformations themselves.  These are
(\G2 $\rightarrow$ \G4, \G4 $\rightarrow$ \G6, \G6 $\rightarrow$ \G2, with
(\G3 $\rightarrow$ \G5, \G5 $\rightarrow$ \G7, \G7 $\rightarrow$ \G3), and
(\G2 $\rightarrow$ \G6, \G6 $\rightarrow$ \G4, \G4 $\rightarrow$ \G2, with
(\G3 $\rightarrow$ \G7, \G7 $\rightarrow$ \G5, \G5 $\rightarrow$ \G3).
There are thus 3 families of transformations based on unsigned permutations
of the group elements.

 In addition we can consistently map \G2 and \G3 to -\G2 and -\G3, provided
we do the same with either the pair \G4 and \G5 or \G6 and \G7.  Likewise
\G4 and \G5 can be mapped to their negatives if we do the same with
\G6 and \G7, giving us 4 possible sign mappings:  the identity and 3 with
4 elements swapping sign.  We have 12 families of transformations.

%%%% A 10-ELEMENT GROUP %%%%%%%%%%%%%%%%%%%%%%%%%%%%%%%%%
\section{ An Example with With a 10-Element Group}
 The non-abelian 10-element group
may be defined by the following equations:
\begin{equation}
a^2 \rightarrow 0 \quad , \quad b^5 \rightarrow 0 \quad , \quad a b \rightarrow
b^4 a
\end{equation}

 We may designate the elements of the group by \G0 , \G1 , \G2 , \G3 , \G4 ,
\G5 , \G6 , \G7, \G8, and \G9 , where \G0 is the identity,
$b \equiv \G1$, $b^2 \equiv \G2$, $b^3 \equiv \G3$, $b^4 \equiv \G4$,
$a \equiv \G5$, $a b \equiv \G6$, $a b^2 \equiv \G7$, $a b^3 \equiv
\G8$, and $a b^4 \equiv \G9$.
We may define a product (Cayley) table for this group in the following way:
\begin{displaymath}
\begin{array}{cccccccccc}
\G0 & \G1 & \G2 & \G3 & \G4 & \G5 & \G6 & \G7 & \G8 & \G9 \\
\G1 & \G2 & \G3 & \G4 & \G0 & \G9 & \G5 & \G6 & \G7 & \G8 \\
\G2 & \G3 & \G4 & \G0 & \G1 & \G8 & \G9 & \G5 & \G6 & \G7 \\
\G3 & \G4 & \G0 & \G1 & \G2 & \G7 & \G8 & \G9 & \G5 & \G6 \\
\G4 & \G0 & \G1 & \G2 & \G3 & \G6 & \G7 & \G8 & \G9 & \G5 \\
\G5 & \G6 & \G7 & \G8 & \G9 & \G0 & \G1 & \G2 & \G3 & \G4 \\
\G6 & \G7 & \G8 & \G9 & \G5 & \G4 & \G0 & \G1 & \G2 & \G3 \\
\G7 & \G8 & \G9 & \G5 & \G6 & \G3 & \G4 & \G0 & \G1 & \G2 \\
\G8 & \G9 & \G5 & \G6 & \G7 & \G2 & \G3 & \G4 & \G0 & \G1 \\
\G9 & \G5 & \G6 & \G7 & \G8 & \G1 & \G2 & \G3 & \G4 & \G0 
\end{array}
\end{displaymath}

 Grinding through the algebra will show that all
elements of $\delta V$ are zero except a few which are described by
six independent tiny changes $\alpha$, $\beta$, $\gamma$, $\epsilon$,
$\mu$, and $\nu$.
\begin{displaymath}
\delta V = \left(
\begin{array}{ccccc}
0 & 0 & 0 & 0 & 0 \\
0 & 0 & 0 & 0 & 0 \\
0 & 0 & 0 & 0 & 0 \\
0 & 0 & 0 & 0 & 0 \\
0 & 0 & 0 & 0 & 0 \\
0 & \nu & -\nu-\epsilon-\gamma & \nu+\epsilon+\gamma & -\nu \\
0 & \mu & \nu+\epsilon & -\nu-\epsilon & -\mu \\
0 & \epsilon & \mu+\gamma & -\mu-\gamma & -\epsilon \\
0 & \gamma & -\nu-\mu-\gamma & \nu+\mu+\gamma & -\gamma \\
0 & -\nu-\mu-\epsilon-\gamma & \nu+\gamma & -\nu-\gamma & 
\nu+\mu+\epsilon+\gamma 
\end{array} \cdots \right.
\end{displaymath}
\begin{displaymath}
\left. \cdots \begin{array}{ccccc}
0 & 0 & 0 & 0 & 0 \\
\nu & \mu & \epsilon & \gamma & -\nu-\mu-\gamma-\epsilon \\
-\nu-\epsilon-\gamma & \nu+\epsilon & \mu+\gamma & -\nu-\mu-\gamma & \nu+\gamma \\
\nu+\epsilon+\gamma & -\nu-\epsilon & -\mu-\gamma & \nu+\mu+\gamma & -\nu-\gamma \\
-\nu & -\mu & -\epsilon & -\gamma & \nu+\mu+\gamma+\epsilon \\
0 & \alpha & \beta & -\beta & -\alpha \\
-\alpha & 0 & \alpha & \beta & -\beta \\
-\beta & -\alpha & 0 & \alpha & \beta \\
\beta & -\beta & -\alpha & 0 & \alpha \\
\alpha & \beta & -\beta & -\alpha & 0 
\end{array} \right)
\end{displaymath}
If we set
\begin{displaymath}
K = \left(
\begin{array}{ccccc}
0 & 1 & 0 & 0 & -1 \\
-1 & 0 & 1 & 0 & 0 \\
0 & -1 & 0 & 1 & 0 \\
0 & 0 & -1 & 0 & 1 \\
1 & 0 & 0 & -1 & 0
\end{array} \right)
\quad , \quad
L = \left(
\begin{array}{ccccc}
0 & 0 & 1 & -1 & 0 \\
0 & 0 & 0 & 1 & -1 \\
-1 & 0 & 0 & 0 & 1 \\
1 & -1 & 0 & 0 & 0 \\
0 & 1 & -1 & 0 & 0
\end{array} \right)
\end{displaymath}
\begin{displaymath}
M = \left(
\begin{array}{ccccc}
0 & 0 & 0 & 0 & 0 \\
1 & 0 & 0 & 0 & -1 \\
-1 & 1 & 0 & -1 & 1 \\
1 & -1 & 0 & 1 & -1 \\
-1 & 0 & 0 & 0 & 1
\end{array} \right)
\quad , \quad
N = \left(
\begin{array}{ccccc}
0 & 0 & 0 & 0 & 0 \\
0 & 1 & 0 & 0 & -1 \\
0 & 0 & 1 & -1 & 0 \\
0 & 0 & -1 & 1 & 0 \\
0 & -1 & 0 & 0 & 1
\end{array} \right)
\end{displaymath}
\begin{displaymath}
P = \left(
\begin{array}{ccccc}
0 & 0 & 0 & 0 & 0 \\
0 & 0 & 1 & 0 & -1 \\
-1 & 1 & 0 & 0 & 0 \\
1 & -1 & 0 & 0 & 0 \\
0 & 0 & -1 & 0 & 1
\end{array} \right)
\quad , \quad
Q = \left(
\begin{array}{ccccc}
0 & 0 & 0 & 0 & 0 \\
0 & 0 & 0 & 1 & -1 \\
-1 & 0 & 1 & -1 & 1 \\
1 & 0 & -1 & 1 & -1 \\
0 & 0 & 0 & -1 & 1
\end{array} \right)
\end{displaymath}

\begin{displaymath}
A = \left( \begin{array}{cc}
0 & 0 \\
0 & K
\end{array} \right) \quad , \quad
B = \left( \begin{array}{cc}
0 & 0 \\
0 & L
\end{array} \right) \quad , \quad
C = \left( \begin{array}{cc}
0 & M \\
M^T & 0
\end{array} \right)
\end{displaymath}

\begin{displaymath}
D = \left( \begin{array}{cc}
0 & N \\
N^T & 0
\end{array} \right) \quad , \quad
E = \left( \begin{array}{cc}
0 & P \\
P^T & 0
\end{array} \right) \quad , \quad
F = \left( \begin{array}{cc}
0 & Q \\
Q^T & 0
\end{array} \right)
\end{displaymath}
The commutators of these generators are:
\begin{eqnarray*}
\left[ A,B \right] = 0 \quad\left[ A,C \right] = C-D-F \quad\left[ A,D \right] = 2C-E-F\\
\left[ A,E \right] = C+D-2F \quad\left[ A,F \right] = C+E-F \quad\left[ B,C \right] = D-2E+F\\
\left[ B,D \right] = D-E-F \quad\left[ B,E \right] = C+D-E \quad\left[ B,F \right] = -C+2D-E\\
\left[ C,D \right] = 2A-2B \quad\left[ C,E \right] = 2B \quad\left[ C,F \right] = 2*A-2B\\
\left[ D,E \right] = 2B \quad\left[ D,F \right] = 2B \quad\left[ E,F \right] = 2A-2B 
\end{eqnarray*}

 The eigenvalues of $\delta V$ are given by the zeros of the
following mess:
\begin{equation}
0 =
x^6 \left(  
\begin{array}{ccccc}
& - X^4 & & & \\
+X^2 \left( \right. &16\gamma\nu &+ 8\gamma\epsilon &+ 8\mu^2 &+ 8\mu\nu \\
+ 4\mu\epsilon &+ 12\nu^2 &+ 12\nu\epsilon &+ 8\epsilon^2 &- 5\alpha^2 \\
- 5\beta^2 &+ 12\gamma^2 &+ \left. 12\gamma\mu \right) & &\\
& & & & \\
+\left( \right.- 112\gamma\nu\epsilon^2 &- 32\gamma\epsilon^3 &- 16\mu^4 
&- 32\mu^3\nu &- 16\mu^3\epsilon \\
- 64\mu^2\nu^2 &- 64\mu^2\nu\epsilon &- 16\mu^2\epsilon^2 
&- 48\mu\nu^3  &- 112\mu\nu^2\epsilon \\
- 96\mu\nu\epsilon^2 &- 16\mu\epsilon^3 &- 16\nu^4 
&- 32\nu^3\epsilon &- 64\nu^2\epsilon^2 \\
- 48\nu\epsilon^3 &- 16\epsilon^4 &- 5\alpha^4 &+ 10\alpha^3\beta  
&+ 5\alpha^2\beta^2 \\
+ 40\alpha^2\gamma^2 &+ 40\alpha^2\gamma\mu &+ 60\alpha^2\gamma\nu  
&+ 20\alpha^2\gamma\epsilon &+ 20\alpha^2\mu^2 \\
+ 20\alpha^2\mu\nu &+ 40\alpha^2\nu^2  &+ 40\alpha^2\nu\epsilon 
&+ 20\alpha^2\epsilon^2 &- 10\alpha\beta^3 \\
+ 40\alpha\beta\gamma^2 &+ 40\alpha\beta\gamma\mu
&+ 80\alpha\beta\gamma\nu &- 40\alpha\beta\mu\epsilon 
&+ 40\alpha\beta\nu^2 \\
+ 40\alpha\beta\nu\epsilon &- 5\beta^4 &+ 20\beta^2\gamma^2  
&+ 20\beta^2\gamma\mu &+ 20\beta^2\gamma\nu \\
+ 20\beta^2\gamma\epsilon &+ 20\beta^2\mu^2  &+ 20\beta^2\mu\nu 
&+ 20\beta^2\mu\epsilon &+ 20\beta^2\nu^2 \\
+ 20\beta^2\nu\epsilon &+ 20\beta^2\epsilon^2 &- 16\gamma^4 
&- 32\gamma^3\mu &- 16\gamma^3\nu \\
- 48\gamma^3\epsilon &- 64\gamma^2\mu^2 &- 64\gamma^2\mu\nu
&- 112\gamma^2\mu\epsilon &- 16\gamma^2\nu^2 \\
- 96\gamma^2\nu\epsilon &- 64\gamma^2\epsilon^2 &- 48\gamma\mu^3
&- 112\gamma\mu^2\nu &- 96\gamma\mu^2\epsilon \\
- 96\gamma\mu\nu^2 &- 176\gamma\mu\nu\epsilon
&- 64\gamma\mu\epsilon^2 &- 16\gamma\nu^3 &- 64\gamma\nu^2\epsilon 
\left. \right)
\end{array}
 \right)
\end{equation}
 If all variables but $\mu$ are zero, then this has real eigenvalues,
but if all but $\alpha$ are zero, then this has four imaginary
eigenvalues.
%%%%%%%%%%%%%%%%%%%%%%%%%% A4 %%%%%%%%%%%%%%%%%%%%%%%%%%%%%%%%%%
\section{A4}

 The alternating group of order 4 (the rotational symmetries of a
tetrahedron) is a rather interesting group.  It's Cayley table is
\begin{displaymath}
\begin{array}{cccccccccccc}
 \G0 &\G1 &\G2 &\G3 &\G4 &\G5 &\G6 &\G7 &\G8 &\G9 &\G1\G0 &\G1\G1 \\
 \G1 &\G0 &\G7 &\G1\G1 &\G6 &\G9 &\G4 &\G2 &\G1\G0 &\G5 &\G8 &\G3 \\
 \G2 &\G5 &\G6 &\G1\G0 &\G8 &\G1\G1 &\G0 &\G3 &\G9 &\G4 &\G7 &\G1 \\
 \G3 &\G4 &\G8 &\G9 &\G7 &\G1\G0 &\G5 &\G1 &\G1\G1 &\G0 &\G6 &\G2 \\
 \G4 &\G3 &\G1 &\G2 &\G5 &\G0 &\G7 &\G8 &\G6 &\G1\G0 &\G1\G1 &\G9 \\
 \G5 &\G2 &\G3 &\G1 &\G0 &\G4 &\G8 &\G6 &\G7 &\G1\G1 &\G9 &\G1\G0 \\
 \G6 &\G1\G1 &\G0 &\G7 &\G9 &\G1 &\G2 &\G1\G0 &\G4 &\G8 &\G3 &\G5 \\
 \G7 &\G9 &\G4 &\G8 &\G1\G0 &\G3 &\G1 &\G1\G1 &\G5 &\G6 &\G2 &\G0 \\
 \G8 &\G1\G0 &\G5 &\G6 &\G1\G1 &\G2 &\G3 &\G9 &\G0 &\G7 &\G1 &\G4 \\
 \G9 &\G7 &\G1\G1 &\G0 &\G1 &\G6 &\G1\G0 &\G4 &\G2 &\G3 &\G5 &\G8 \\
 \G1\G0 &\G8 &\G9 &\G4 &\G3 &\G7 &\G1\G1 &\G5 &\G1 &\G2 &\G0 &\G6 \\
 \G1\G1 &\G6 &\G1\G0 &\G5 &\G2 &\G8 &\G9 &\G0 &\G3 &\G1 &\G4 &\G7
\end{array}
\end{displaymath}

 After grinding through the algebra one finds for it's differential
matrix:
\begin{displaymath}
\begin{array}{ccccccc}
0 & 0 & 0 & 0 & 0 & 0 & ... \\
0 & 0 & \tau & \nu & \epsilon & -\beta & ... \\
0 & \nu & 0 & \tau & \delta-\tau& \gamma & ... \\
0 & \tau & \nu & 0 & -\alpha & \epsilon+\mu & ... \\
0 & \beta & \mu & -\alpha & 0 & 0 & ... \\
0 & -\epsilon & \gamma & \beta+\delta-\tau & 0 & 0 & ... \\
0 & -\tau & 0 & \alpha+\gamma & -\gamma & -\mu & ... \\
0 & \epsilon & \alpha & -\beta-\delta & \beta & -\gamma-\alpha &... \\
0 & 0 & -\delta & \epsilon+\mu & -\epsilon-\mu & \beta+\delta-\tau & ... \\
0 & -\nu & -\gamma-\alpha & 0 & \tau-\beta-\delta & \alpha & ... \\
0 & 0 & \delta-\tau & -\epsilon-\mu-\nu & \mu & \tau-\delta & ... \\
0 & -\beta & -\mu-\nu & -\gamma & \gamma+\alpha & -\epsilon & ... \\
& & & & & & \\
... & 0 & 0 & 0 & 0 & 0 & 0 \\
... & -\nu & \beta & 0 & -\tau & 0 & -\epsilon \\
... & 0 & \alpha & -\nu-\mu & -\alpha-\gamma & \mu & -\delta \\
...& \alpha+\gamma & -\epsilon-\nu-\mu & \beta+\delta-\tau & 0 & -\beta-\delta & -\gamma \\
... & -\gamma & \epsilon & \tau-\beta-\delta & -\epsilon-\mu & \delta-\tau & \alpha+\gamma \\
...&  \tau-\delta & -\alpha-\gamma& \epsilon+\mu & \alpha & -\mu & -\beta \\
... & 0 & \mu+\nu & \delta & -\nu & \tau-\delta & -\alpha \\
... & \delta & 0 & \mu+\nu & \gamma & -\epsilon-\mu-\nu & 0 \\
... & \mu+\nu & \delta & 0 & \tau-\beta-\delta & 0 & -\mu-\nu \\
... & -\tau & \gamma & -\epsilon-\mu & 0 & \epsilon+\mu+\nu & \beta+\delta \\
... & -\mu & -\beta-\delta & 0 & \beta+\delta & 0 & \epsilon+\mu+\nu \\
... &  -\alpha & 0 & -\delta & \epsilon+\mu+\nu & \beta+\delta & 0 
\end{array}
\end{displaymath}

 There are 8 generators that appear from the above, whose commutation relations
look like
\begin{displaymath}
\begin{array}{cc}
\left[ A,B \right] = C - 2F -H &
\left[ A,C \right] = B - 2E -G \\
\left[ A,D \right] = 0 &
\left[ A,E \right] = -C -H \\
\left[ A,F \right] = -B -G &
\left[ A,G \right] = C -2F -H \\
\left[ A,H \right] = B -2E -G &
\left[ B,C \right] = 0 \\
\left[ B,D \right] = -C +F +2H &
\left[ B,E \right] = C-F-2H \\
\left[ B,F \right] = 4A-B-2D+E+2G &
\left[ B,G \right] = -C+2F+H \\
\left[ B,H \right] = -2A+B+4D-2E-G &
\left[ C,D \right] = -B+E+2G \\
\left[ C,E \right] = -4A-B+2D+E+2G &
\left[ C,F \right] = C-F-2H  \\
\left[ C,G \right] = 2A+B-4D-2E-G &
\left[ C,H \right] = -C+2F+H \\
\left[ D,E \right] = C-F-2H &
\left[ D,F \right] = B-E-2G \\
\left[ D,G \right] = -C-F &
\left[ D,H \right] = -B-E \\
\left[ E,F \right] = 0 &
\left[ E,G \right] = -2C+F+H  \\
\left[ E,H \right] = 2A+2B+2D-E-G &
\left[ F,G \right] = -2A+2B-2D-E-G \\
\left[ F,H \right] = -2C+F+H &
\left[ G,H \right] = 0
\end{array}
\end{displaymath}

	This is certainly messy, but some simplifications help.

%%%% 16-ELEMENT GROUP Number 3 Table 22 %%%%%%%%%%%%%%%%%%%%%%%%%%%%%%%%%
\section{ Examples With 16-Element Groups}
\subsection{16-element group 1}

%%% 16_1.DAT, Table 20 %%%%%%%%%%%%%%%%%%%%%%%%%%%%%%%%%%%%%%%%
 One of the non-abelian 16-element groups may be defined as in the
following table with the group elements designated by \G0, \G1,
\G2 . . . \G1\G5.  It generated 
by the relations $a^2 = 0$, $b^2 = a$, $c^2 = b^{-1}$, $d^2 = 0$,
$a = b^{-1}d^{-1}bd$, and $b=c^{-1}d^{-1}cd$; where
$a$ is \G4, $b$ is \G2, $c$ is \G3, and
$d$ is \G8.  The resulting product (Cayley) table is:
\begin{displaymath}
\begin{array}{cccccccccccccccc}
\G0 &\G1 &\G2 &\G3 &\G4 &\G5 &\G6 &\G7 &\G8 &\G9 &\G1\G0 &\G1\G1 &\G1\G2 &\G1\G3 &\G1\G4 &\G1\G5 \\
\G1 &\G2 &\G3 &\G4 &\G5 &\G6 &\G7 &\G0 &\G1\G4 &\G1\G5 &\G1\G3 &\G1\G2 &\G8 &\G9 &\G1\G0 &\G1\G1 \\
\G2 &\G3 &\G4 &\G5 &\G6 &\G7 &\G0 &\G1 &\G1\G0 &\G1\G1 &\G9 &\G8 &\G1\G4 &\G1\G5 &\G1\G3 &\G1\G2 \\
\G3 &\G4 &\G5 &\G6 &\G7 &\G0 &\G1 &\G2 &\G1\G3 &\G1\G2 &\G1\G5 &\G1\G4 &\G1\G0 &\G1\G1 &\G9 &\G8 \\
\G4 &\G5 &\G6 &\G7 &\G0 &\G1 &\G2 &\G3 &\G9 &\G8 &\G1\G1 &\G1\G0 &\G1\G3 &\G1\G2 &\G1\G5 &\G1\G4 \\
\G5 &\G6 &\G7 &\G0 &\G1 &\G2 &\G3 &\G4 &\G1\G5 &\G1\G4 &\G1\G2 &\G1\G3 &\G9 &\G8 &\G1\G1 &\G1\G0 \\
\G6 &\G7 &\G0 &\G1 &\G2 &\G3 &\G4 &\G5 &\G1\G1 &\G1\G0 &\G8 &\G9 &\G1\G5 &\G1\G4 &\G1\G2 &\G1\G3 \\
\G7 &\G0 &\G1 &\G2 &\G3 &\G4 &\G5 &\G6 &\G1\G2 &\G1\G3 &\G1\G4 &\G1\G5 &\G1\G1 &\G1\G0 &\G8 &\G9 \\
\G8 &\G1\G2 &\G1\G1 &\G1\G5 &\G9 &\G1\G3 &\G1\G0 &\G1\G4 &\G0 &\G4 &\G6 &\G2 &\G1 &\G5 &\G7 &\G3 \\
\G9 &\G1\G3 &\G1\G0 &\G1\G4 &\G8 &\G1\G2 &\G1\G1 &\G1\G5 &\G4 &\G0 &\G2 &\G6 &\G5 &\G1 &\G3 &\G7 \\
\G1\G0 &\G1\G4 &\G8 &\G1\G2 &\G1\G1 &\G1\G5 &\G9 &\G1\G3 &\G2 &\G6 &\G0 &\G4 &\G3 &\G7 &\G1 &\G5 \\
\G1\G1 &\G1\G5 &\G9 &\G1\G3 &\G1\G0 &\G1\G4 &\G8 &\G1\G2 &\G6 &\G2 &\G4 &\G0 &\G7 &\G3 &\G5 &\G1 \\
\G1\G2 &\G1\G1 &\G1\G5 &\G9 &\G1\G3 &\G1\G0 &\G1\G4 &\G8 &\G7 &\G3 &\G5 &\G1 &\G0 &\G4 &\G6 &\G2 \\
\G1\G3 &\G1\G0 &\G1\G4 &\G8 &\G1\G2 &\G1\G1 &\G1\G5 &\G9 &\G3 &\G7 &\G1 &\G5 &\G4 &\G0 &\G2 &\G6 \\
\G1\G4 &\G8 &\G1\G2 &\G1\G1 &\G1\G5 &\G9 &\G1\G3 &\G1\G0 &\G1 &\G5 &\G7 &\G3 &\G2 &\G6 &\G0 &\G4 \\
\G1\G5 &\G9 &\G1\G3 &\G1\G0 &\G1\G4 &\G8 &\G1\G2 &\G1\G1 &\G5 &\G1 &\G3 &\G7 &\G6 &\G2 &\G4 &\G0
\end{array}
\end{displaymath}
 This group has a set of infinitesimal transformations given by
\begin{displaymath}
\delta V = \left(
\begin{array}{cccccccc}
0 &   0 &  0 &  0  & 0 &  0  &  0 &  0  \\ 
0 &   0 &  0 &  0  & 0 &  0  &  0 &  0  \\ 
0 &   0 &  0 &  0  & 0 &  0  &  0 &  0  \\
0 &   0 &  0 &  0  & 0 &  0  &  0 &  0  \\
0 &   0 &  0 &  0  & 0 &  0  &  0 &  0  \\
0 &   0 &  0 &  0  & 0 &  0  &  0 &  0  \\ 
0 &   0 &  0 &  0  & 0 &  0  &  0 &  0  \\
0 &   0 &  0 &  0  & 0 &  0  &  0 &  0  \\
0 &   \lambda & \mu-\epsilon& -\rho  & 0 &  \rho  & \epsilon-\mu& -\lambda  \\ 
0 &   \rho & \epsilon-\mu& -\lambda  & 0 &  \lambda  & \mu-\epsilon& -\rho  \\
0 &  -\sigma & \nu-\epsilon&\lambda+\rho-\sigma& 0 &\sigma-\rho-\lambda& \epsilon-\nu&  \sigma  \\
0 &\sigma-\rho-\lambda& \epsilon-\nu&  \sigma  & 0 & -\sigma  & \nu-\epsilon&\lambda+\rho-\sigma \\
0 &   \mu & \sigma-\rho& -\nu  & 0 &  \nu  & \rho-\sigma& -\mu  \\ 
0 &   \nu & \rho-\sigma& -\mu  & 0 &  \mu  & \sigma-\rho& -\nu  \\
0 &  -\epsilon & \lambda-\sigma&\nu+\mu-\epsilon& 0 &\epsilon-\nu-\mu& \sigma-\lambda&  \epsilon  \\ 
0 &\epsilon-\mu-\nu& \sigma-\lambda&  \epsilon  & 0 & -\epsilon  & \lambda-\sigma&\mu+\nu-\epsilon
\end{array} \right. \cdots
\end{displaymath}
\begin{displaymath}
\cdots \left.
\begin{array}{cccccccc}
0 &  0 &  0  &   0 &  0 &  0 &   0 &   0 \\
\lambda &  \rho & -\sigma  &\sigma-\rho-\lambda&  \mu &  \nu &  -\epsilon & \epsilon-\mu-\nu \\
\mu-\epsilon &\epsilon-\mu & \nu-\epsilon & \epsilon-\nu &\sigma-\rho &\rho-\sigma & \lambda-\sigma &  \sigma-\lambda \\
-\rho & -\lambda &\rho+\lambda-\sigma&   \sigma & -\nu & -\mu &\mu+\nu-\epsilon&   \epsilon \\
0 &  0 &  0  &   0 &  0 &  0 &   0 &   0 \\
\rho &  \lambda &\sigma-\rho-\lambda&  -\sigma &  \nu &  \mu &\epsilon-\mu-\nu&  -\epsilon \\
\epsilon-\mu &\mu-\epsilon & \epsilon-\nu & \nu-\epsilon &\rho-\sigma &\sigma-\rho & \sigma-\lambda &  \lambda-\sigma \\
-\lambda & -\rho &  \sigma  &\lambda+\rho-\sigma& -\mu & -\nu &   \epsilon & \mu+\nu-\epsilon \\
0 &  0 & -\alpha  &   \alpha & -\beta & -\gamma &   \beta &   \gamma \\
0 &  0 &  \alpha  & - \alpha & -\gamma & -\beta &   \gamma &   \beta \\
\alpha & -\alpha &  0  &   0 &  \gamma &  \beta &  -\beta &  -\gamma \\
-\alpha &  \alpha &  0  &   0 &  \beta &  \gamma &  -\gamma &  -\beta \\
\beta &  \gamma & -\gamma  &  -\beta &  0 &  0 &  -\alpha &   \alpha \\
\gamma &  \beta & -\beta  &  -\gamma &  0 &  0 &   \alpha &  -\alpha \\
-\beta & -\gamma &  \beta  &   \gamma &  \alpha & -\alpha &   0 &   0 \\
-\gamma & -\beta &  \gamma  &   \beta & -\alpha &  \alpha &   0 &   0
\end{array} \right)
\end{displaymath}

 Here $\delta V$ is given by 
$\alpha A' + \beta B' + \gamma G' + \mu M' + \nu N' + \rho R' + \sigma S' +
\lambda L' + \epsilon E'$.  If we combine these we can simplify the
commutation relations.  For instance, let $E = M' + N' + E'$, $L = R'+S'+L'$,
and $G = G' + B'$.  Then the 36 non-trivial commutation relations appear
as
\begin{eqnarray*}
\left[ A,B \right] = 0 & \left[ A,G \right] = 0 & \left[ A,L \right] = 0\\
\left[ A,E \right] = 0 & \left[ B,G \right] = 0 & \left[ G,S \right] = 0\\
\left[ M,E \right] = 0 & \left[ N,S \right] = 0 & \left[ N,E \right] = 0\\
\left[ R,L \right] = 0 & \left[ S,L \right] = 0 & \left[ S,E \right] = 0\\
\left[ A,M \right] = E-2N & \left[ A,N \right] = 2M-E & \left[ A,R \right] = L2R-2S \\
\left[ A,S \right] = 4R+2S-2L & \left[ B,M \right] = L-S & \left[ B,N \right] = 2R+S\\
\left[ B,R \right] = -M-N & \left[ B,S \right] = 2M-ED & \left[ B,L \right] = -2E\\
\left[ B,E \right] = 2L & \left[ G,M \right] = 2L & \left[ G,N \right] = 2L\\
\left[ G,R \right] = -2E & \left[ G,L \right] = -4E & \left[ G,E \right] = 4L\\
\left[ M,N \right] = 2A & \left[ M,R \right] = -2B & \left[ M,S \right] = 4B-2G\\
\left[ M,L \right] = -2G & \left[ N,R \right] = -2B & \left[ N,L \right] = -2G\\
\left[ R,S \right] = 2A & \left[ R,E \right] = 2G & \left[ L,E \right] = 4G
\end{eqnarray*}

 The generators $A$, $B$, and $G$ form one subset.  Note that the commutators
of $G$, $L$, and $E$ cycle, as do those of $A$, $E-2M$, and $E-2N$.  At the
moment I am unable to identify this with one of the classical Lie groups.

 The eigenfunction of the differential tranformation array is rather
messy, but has the form
$$
0 = x^{8} \left( x^6 + \left( \right) x^4 + \left( \right) x^2 +
\left( \right) \right)
$$
where the last quantity in parenthesis is about 2 pages long.  The
quantities inside the parentheses are polynomials in the differential
changes.

\subsection{16-element group 2}

%%% 16_2.DAT, Table 21 %%%%%%%%%%%%%%%%%%%%%%%%%%%%%%%%%%%%%%%%
 Another one of the non-abelian 16-element groups may be defined as in the
following table with the group elements designated by \G0, \G1,
\G2 . . . \G1\G5.  It generated 
by the relations $a^2 = b^2 = c^2 = 0$, $x^2 = a$, and
$a = b^{-1}c^{-1}bc$, where $x$ is \G1, $a$ is \G2, $b$ is \G9, and
$c$ is \G1\G3.  The resulting product (Cayley) table is:
\begin{displaymath}
\begin{array}{cccccccccccccccc}
\G0 & \G1 & \G2 & \G3 & \G4 & \G5 & \G6 & \G7 & \G8 & \G9 & \G1\G0 & \G1\G1 & 
\G1\G2 & \G1\G3 & \G1\G4 & \G1\G5\\
\G1 & \G2 & \G3 & \G0 & \G5 & \G6 & \G7 & \G4 & \G9 & \G1\G0 &
 \G1\G1 & \G8 & \G1\G3 & \G1\G4 & \G1\G5 & \G1\G2\\
\G2 & \G3 & \G0 & \G1 & \G6 & \G7 & \G4 & \G5 & \G1\G0 & \G1\G1 &
 \G8 & \G9 & \G1\G4 & \G1\G5 & \G1\G2 & \G1\G3\\
\G3 & \G0 & \G1 & \G2 & \G7 & \G4 & \G5 & \G6 & \G1\G1 & \G8 & \G9 &
 \G1\G0 & \G1\G5 & \G1\G2 & \G1\G3 & \G1\G4\\
\G4 & \G5 & \G6 & \G7 & \G2 & \G3 & \G0 & \G1 & \G1\G4 & \G1\G5 &
 \G1\G2 & \G1\G3 & \G8 & \G9 & \G1\G0 & \G1\G1\\
\G5 & \G6 & \G7 & \G4 & \G3 & \G0 & \G1 & \G2 & \G1\G5 & \G1\G2 & 
\G1\G3 & \G1\G4 & \G9 & \G1\G0 & \G1\G1 & \G8\\
\G6 & \G7 & \G4 & \G5 & \G0 & \G1 & \G2 & \G3 & \G1\G2 & \G1\G3 &
 \G1\G4 & \G1\G5 & \G1\G0 & \G1\G1 & \G8 & \G9\\
\G7 & \G4 & \G5 & \G6 & \G1 & \G2 & \G3 & \G0 & \G1\G3 & \G1\G4 &
 \G1\G5 & \G1\G2 & \G1\G1 & \G8 & \G9 & \G1\G0\\
\G8 & \G9 & \G1\G0 & \G1\G1 & \G1\G2 & \G1\G3 & \G1\G4 & \G1\G5 &
 \G2 & \G3 & \G0 & \G1 & \G6 & \G7 & \G4 & \G5\\
\G9 & \G1\G0 & \G1\G1 & \G8 & \G1\G3 & \G1\G4 & \G1\G5 & \G1\G2 &
 \G3 & \G0 & \G1 & \G2 & \G7 & \G4 & \G5 & \G6\\
\G1\G0 & \G1\G1 & \G8 & \G9 & \G1\G4 & \G1\G5 & \G1\G2 & \G1\G3 &
 \G0 & \G1 & \G2 & \G3 & \G4 & \G5 & \G6 & \G7\\
\G1\G1 & \G8 & \G9 & \G1\G0 & \G1\G5 & \G1\G2 & \G1\G3 & \G1\G4 &
 \G1 & \G2 & \G3 & \G0 & \G5 & \G6 & \G7 & \G4\\
\G1\G2 & \G1\G3 & \G1\G4 & \G1\G5 & \G1\G0 & \G1\G1 & \G8 & \G9 &
 \G4 & \G5 & \G6 & \G7 & \G2 & \G3 & \G0 & \G1\\
\G1\G3 & \G1\G4 & \G1\G5 & \G1\G2 & \G1\G1 & \G8 & \G9 & \G1\G0 &
 \G5 & \G6 & \G7 & \G4 & \G3 & \G0 & \G1 & \G2\\
\G1\G4 & \G1\G5 & \G1\G2 & \G1\G3 & \G8 & \G9 & \G1\G0 & \G1\G1 &
 \G6 & \G7 & \G4 & \G5 & \G0 & \G1 & \G2 & \G3\\
\G1\G5 & \G1\G2 & \G1\G3 & \G1\G4 & \G9 & \G1\G0 & \G1\G1 & \G8 &
 \G7 & \G4 & \G5 & \G6 & \G1 & \G2 & \G3 & \G0
\end{array}
\end{displaymath}
 This group has a set of infinitesimal transformations given by
\begin{displaymath}
\begin{array}{cccccccccccccccc}
0 & 0 & 0 & 0 & 0 & 0 & 0 & 0 & 0 & 0 & 0 & 0 & 0 & 0 & 0 & 0 \\
0 & 0 & 0 & 0 & 0 & 0 & 0 & 0 & 0 & 0 & 0 & 0 & 0 & 0 & 0 & 0 \\
0 & 0 & 0 & 0 & 0 & 0 & 0 & 0 & 0 & 0 & 0 & 0 & 0 & 0 & 0 & 0 \\
0 & 0 & 0 & 0 & 0 & 0 & 0 & 0 & 0 & 0 & 0 & 0 & 0 & 0 & 0 & 0 \\
0 & 0 & 0 & 0 & 0 & 0 & 0 & 0 & \epsilon & \mu & -\epsilon & -\mu &
\gamma & -\delta & -\gamma & \delta \\
0 & 0 & 0 & 0 & 0 & 0 & 0 & 0 & -\mu & \epsilon & \mu & -\epsilon &
\delta & \gamma & -\delta & -\gamma \\
0 & 0 & 0 & 0 & 0 & 0 & 0 & 0 & -\epsilon & -\mu & \epsilon & \mu &
-\gamma & \delta & \gamma & -\delta \\
0 & 0 & 0 & 0 & 0 & 0 & 0 & 0 & \mu & -\epsilon & -\mu & \epsilon &
-\delta & -\gamma & \delta & \gamma \\
0 & 0 & 0 & 0 & -\epsilon & -\mu & \epsilon & \mu & 0 & 0 & 0 & 0 & 
-\alpha & \beta & \alpha & -\beta \\
0 & 0 & 0 & 0 & \mu & -\epsilon & -\mu & \epsilon & 0 & 0 & 0 & 0 & 
-\beta & -\alpha & \beta & \alpha \\
0 & 0 & 0 & 0 & \epsilon & \mu & -\epsilon & -\mu & 0 & 0 & 0 & 0 & 
\alpha & -\beta & -\alpha & \beta \\
0 & 0 & 0 & 0 & -\mu & \epsilon & \mu & -\epsilon & 0 & 0 & 0 & 0 & 
\beta & \alpha & -\beta & -\alpha \\
0 & 0 & 0 & 0 & -\gamma & \delta & \gamma & -\delta &
\alpha & -\beta & -\alpha & \beta & 0 & 0 & 0 & 0 \\
0 & 0 & 0 & 0 & -\delta & -\gamma & \delta & \gamma &
\beta & \alpha & -\beta & -\alpha & 0 & 0 & 0 & 0 \\
0 & 0 & 0 & 0 & \gamma & -\delta & -\gamma & \delta &
-\alpha & \beta & \alpha & -\beta & 0 & 0 & 0 & 0 \\
0 & 0 & 0 & 0 & \delta & \gamma & -\delta & -\gamma & 
-\beta & -\alpha & \beta & \alpha & 0 & 0 & 0 & 0 
\end{array}
\end{displaymath}
If we define S and T such that
$$
S \equiv {1 \over 2}  \left( 
\begin{array}{cccc}
1 & 0 & -1 & 0 \\
0 & 1 & 0 & -1 \\
-1 & 0 & 1 & 0 \\
0 & -1 & 0 & 1
\end{array} \right) \quad
T \equiv {1 \over 2}  \left(
\begin{array}{cccc}0 & 1 & 0 & -1 \\
-1 & 0 & 1 & 0 \\
0 & -1 & 0 & 1 \\
1 & 0 & -1 & 0
\end{array} \right)
$$
\begin{eqnarray*}
A = \left( \begin{array}{cccc}
0 & 0 & 0 & 0 \\
0 & 0 & 0 & 0 \\
0 & 0 & 0 & -S \\
0 & 0 & S & 0
\end{array} \right) \quad
B = \left( \begin{array}{cccc}
0 & 0 & 0 & 0 \\
0 & 0 & 0 & 0 \\
0 & 0 & 0 & T \\
0 & 0 & -T & 0
\end{array} \right) \\		% first line
C = \left( \begin{array}{cccc}
0 & 0 & 0 & 0 \\
0 & 0 & 0 & S \\
0 & 0 & 0 & 0 \\
0 & -S & 0 & 0 
\end{array} \right) \quad
D = \left( \begin{array}{cccc}
0 & 0 & 0 & 0 \\
0 & 0 & 0 & T \\
0 & 0 & 0 & 0 \\
0 & -T & 0 & 0
\end{array} \right) \\
E = \left( \begin{array}{cccc}
0 & 0 & 0 & 0 \\
0 & 0 & S & 0 \\
0 & -S & 0 & 0 \\
0 & 0 & 0 & 0  
\end{array} \right) \quad
F = \left( \begin{array}{cccc}
0 & 0 & 0 & 0 \\
0 & 0 & T & 0 \\
0 & -T & 0 & 0 \\
0 & 0 & 0 & 0  
\end{array} \right)
\end{eqnarray*}
Notice that $S^2 = S$, $T^2=-S$, and $ST = TS = T$.  These two arrays are
thus isomorphic to $1$ and $i$.  The commutation relations among the
generators are:
\begin{eqnarray*}
\left[ A,B \right] =  0 \quad\left[ C,D \right] = 0 \quad\left[ E,F \right] = 0\\
\left[ C,E \right] = -A \quad\left[ E,A \right] = -C \quad\left[ A,C \right] = -E\\
\left[ D,F \right] =  A \quad\left[ F,A \right] = -D \quad\left[ A,D \right] = -F\\
\left[ D,E \right] =  B \quad\left[ E,B \right] =  D \quad\left[ B,D \right] = -E\\
\left[ C,F \right] =  B \quad\left[ F,B \right] = -C \quad\left[ B,C \right] =  F
\end{eqnarray*}
These are the generators of the special group Sl(2,c), the set of 2x2 complex
matrices with determinant 1?
This is NOT isomorphic to the previous group.
\subsection{16-element group 3}
 One of the non-abelian 16-element groups may be defined as in the
following table with the group elements designated by 
The relations $a^2 = b^2 = x$, $x = a^{-1}b^{-1}ab$, and $x^2 = y^2 =0$
define a non-abelian 16-element group, with group elements I will designate
\G0, \G1, \G2 . . . \G1\G5, where $a$ is \G8, $b$ is \G1\G2, $x$ is \G1,
and $y$ is \G2.  This gives the following product (Cayley) table:
\begin{equation}
\begin{array}{cccccccccccccccc}
\G0 & \G1 & \G2 & \G3 & \G4 & \G5 & \G6 & \G7 & \G8 & \G9 & \G1\G0 & \G1\G1 & 
\G1\G2 & \G1\G3 & \G1\G4 & \G1\G5\\
\G1 & \G0 & \G3 & \G2 & \G5 & \G4 & \G7 & \G6 & \G9 & \G8 & \G1\G1 & \G1\G0 & 
\G1\G3 & \G1\G2 & \G1\G5 & \G1\G4\\
\G2 & \G3 & \G0 & \G1 & \G6 & \G7 & \G4 & \G5 & \G1\G0 & \G1\G1 & \G8 & \G9 & 
\G1\G4 & \G1\G5 & \G1\G2 & \G1\G3\\
\G3 & \G2 & \G1 & \G0 & \G7 & \G6 & \G5 & \G4 & \G1\G1 & \G1\G0 & \G9 & \G8 & 
\G1\G5 & \G1\G4 & \G1\G3 & \G1\G2\\
\G4 & \G5 & \G6 & \G7 & \G1 & \G0 & \G3 & \G2 & \G1\G2 & \G1\G3 & \G1\G4 & \G1\G5 & 
\G9 & \G8 & \G1\G1 & \G1\G0\\
\G5 & \G4 & \G7 & \G6 & \G0 & \G1 & \G2 & \G3 & \G1\G3 & \G1\G2 & \G1\G5 & \G1\G4 & 
\G8 & \G9 & \G1\G0 & \G1\G1\\
\G6 & \G7 & \G4 & \G5 & \G3 & \G2 & \G1 & \G0 & \G1\G4 & \G1\G5 & \G1\G2 & \G1\G3 & 
\G1\G1 & \G1\G0 & \G9 & \G8\\
\G7 & \G6 & \G5 & \G4 & \G2 & \G3 & \G0 & \G1 & \G1\G5 & \G1\G4 & \G1\G3 & \G1\G2 & 
\G1\G0 & \G1\G1 & \G8 & \G9\\
\G8 & \G9 & \G1\G0 & \G1\G1 & \G1\G3 & \G1\G2 & \G1\G5 & \G1\G4 & \G1 & \G0 & \G3 & \G2 & 
\G4 & \G5 & \G6 & \G7\\
\G9 & \G8 & \G1\G1 & \G1\G0 & \G1\G2 & \G1\G3 & \G1\G4 & \G1\G5 & \G0 & \G1 & \G2 & \G3 & 
\G5 & \G4 & \G7 & \G6\\
\G1\G0 & \G1\G1 & \G8 & \G9 & \G1\G5 & \G1\G4 & \G1\G3 & \G1\G2 & \G3 & \G2 & \G1 & \G0 & 
\G6 & \G7 & \G4 & \G5\\
\G1\G1 & \G1\G0 & \G9 & \G8 & \G1\G4 & \G1\G5 & \G1\G2 & \G1\G3 & \G2 & \G3 & \G0 & \G1 & 
\G7 & \G6 & \G5 & \G4\\
\G1\G2 & \G1\G3 & \G1\G4 & \G1\G5 & \G8 & \G9 & \G1\G0 & \G1\G1 & \G5 & \G4 & \G7 & \G6 & 
\G1 & \G0 & \G3 & \G2\\
\G1\G3 & \G1\G2 & \G1\G5 & \G1\G4 & \G9 & \G8 & \G1\G1 & \G1\G0 & \G4 & \G5 & \G6 & \G7 & 
\G0 & \G1 & \G2 & \G3\\
\G1\G4 & \G1\G5 & \G1\G2 & \G1\G3 & \G1\G0 & \G1\G1 & \G8 & \G9 & \G7 & \G6 & \G5 & \G4 & 
\G3 & \G2 & \G1 & \G0\\
\G1\G5 & \G1\G4 & \G1\G3 & \G1\G2 & \G1\G1 & \G1\G0 & \G9 & \G8 & \G6 & \G7 & \G4 & \G5 & 
\G2 & \G3 & \G0 & \G1
\end{array}
\end{equation}
 Using the standard techniques, we find that $\delta$ V depends on 6
parameters as follows:
\begin{equation}
\delta V = \left(
\begin{array}{cccccccccccccccc}
0 & 0 & 0 & 0 & 0 & 0 & 0 & 0 & 0 & 0 & 0 & 0 & 0 & 0 & 0 & 0 \\
0 & 0 & 0 & 0 & 0 & 0 & 0 & 0 & 0 & 0 & 0 & 0 & 0 & 0 & 0 & 0 \\
0 & 0 & 0 & 0 & 0 & 0 & 0 & 0 & 0 & 0 & 0 & 0 & 0 & 0 & 0 & 0 \\
0 & 0 & 0 & 0 & 0 & 0 & 0 & 0 & 0 & 0 & 0 & 0 & 0 & 0 & 0 & 0 \\
0 & 0 & 0 & 0 & 0 & 0 & 0 & 0 & \epsilon & -\epsilon & \mu & -\mu
& \gamma & -\gamma & \delta & -\delta \\
0 & 0 & 0 & 0 & 0 & 0 & 0 & 0 & -\epsilon & \epsilon & -\mu & \mu
& -\gamma & \gamma & -\delta & \delta \\
0 & 0 & 0 & 0 & 0 & 0 & 0 & 0 & \mu & -\mu & \epsilon & -\epsilon
& \delta & -\delta & \gamma & -\gamma \\
0 & 0 & 0 & 0 & 0 & 0 & 0 & 0 & -\mu & \mu & -\epsilon & \epsilon
& -\delta & \delta & -\gamma & \gamma \\
0 & 0 & 0 & 0 & -\epsilon & \epsilon & -\mu & \mu & 0 & 0 & 0 & 0 & 
\alpha & -\alpha & \beta & -\beta \\
0 & 0 & 0 & 0 & \epsilon & -\epsilon & \mu & -\mu & 0 & 0 & 0 & 0 & 
-\alpha & \alpha & -\beta & \beta \\
0 & 0 & 0 & 0 & -\mu & \mu & -\epsilon & \epsilon & 0 & 0 & 0 & 0 & 
\beta & -\beta & \alpha & -\alpha \\
0 & 0 & 0 & 0 & \mu & -\mu & \epsilon & -\epsilon & 0 & 0 & 0 & 0 & 
-\beta & \beta & -\alpha & \alpha \\
0 & 0 & 0 & 0 & -\gamma & \gamma & -\delta & \delta & -\alpha & \alpha &
-\beta & \beta & 0 & 0 & 0 & 0 \\
0 & 0 & 0 & 0 & \gamma & -\gamma & \delta & -\delta & \alpha & -\alpha &
\beta & -\beta & 0 & 0 & 0 & 0 \\
0 & 0 & 0 & 0 & -\delta & \delta & -\gamma & \gamma & -\beta & \beta &
-\alpha & \alpha & 0 & 0 & 0 & 0 \\
0 & 0 & 0 & 0 & \delta & -\delta & \gamma & -\gamma & \beta & -\beta &
\alpha & -\alpha & 0 & 0 & 0 & 0
\end{array} \right) 
\end{equation}
A simple pattern is immediately evident.  Let
\begin{equation}
Q \equiv  {1 \over 2} \left(
\begin{array}{cc}
1 & -1 \\
-1 & 1
\end{array}
\right) \quad
R = \left(
\begin{array}{cc}
Q & 0 \\
0 & Q
\end{array}
\right) \quad
S = \left(
\begin{array}{cc}
0 & Q \\
Q & 0
\end{array}
\right)
\end{equation}
 It is fairly easy to see that $RS = SR = -S$ and that $R^2 = S^2 = -R$.
Now define the `infinitesimal generators' in our space, A through F as
\begin{eqnarray*}
A = \left( \begin{array}{cccc}
0 & 0 & 0 & 0 \\
0 & 0 & 0 & 0 \\
0 & 0 & 0 & -R \\
0 & 0 & R & 0
\end{array} \right) \quad
B = \left( \begin{array}{cccc}
0 & 0 & 0 & 0 \\
0 & 0 & 0 & 0 \\
0 & 0 & 0 & -S \\
0 & 0 & S & 0
\end{array} \right) \\		% first line
C = \left( \begin{array}{cccc}
0 & 0 & 0 & 0 \\
0 & 0 & 0 & -R \\
0 & 0 & 0 & 0 \\
0 & R & 0 & 0
\end{array} \right) \quad
D = \left( \begin{array}{cccc}
0 & 0 & 0 & 0 \\
0 & 0 & 0 & -S \\
0 & 0 & 0 & 0 \\
0 & S & 0 & 0
\end{array} \right) \\		% second line
E = \left( \begin{array}{cccc}
0 & 0 & 0 & 0 \\
0 & 0 & -R & 0 \\
0 & R & 0 & 0 \\
0 & 0 & 0 & 0
\end{array} \right) \quad
F = \left( \begin{array}{cccc}
0 & 0 & 0 & 0 \\
0 & 0 & -S & 0 \\
0 & S & 0 & 0 \\
0 & 0 & 0 & 0
\end{array} \right)
\end{eqnarray*}
The commutators of these generators are very simple:
\begin{eqnarray*}
\left[ A,B \right] = 0 \quad\left[ C,D \right] = 0 \quad\left[ E,F \right] = 0\\
\left[ C,E \right] = A \quad\left[ E,A \right] = C \quad\left[ A,C \right] = E \\
\left[ D,F \right] = A \quad\left[ F,A \right] = D \quad\left[ A,D \right] = F \\
\left[ C,F \right] = B \quad\left[ F,B \right] = C \quad\left[ B,C \right] = F\\
\left[ D,E \right] = B \quad\left[ E,B \right] = D \quad\left[ B,D \right] = E 
\end{eqnarray*}
Using the notation of Gilmore \footnote{\underline{Lie Groups, Lie Algebras, 
and Some of Their Applications}, page 187},
if I make the identifications
\begin{eqnarray*}
A = O_{13} \quad B = O_{24} \quad C = O_{12} \quad D = -O_{34}
\quad E = O_{23} \quad F = -O_{14}
\end{eqnarray*}
these are just the generators of the group SO(4).
This is NOT isomorphic to the previous group.

\subsection{16-element group 4}

%  16_4

 Another of the 16-element groups is generated by the elements
$\G3^2=\G1\G1$,  $\G8^2=\G0$, $\G1\G1=\G3^{-1}\G8^{-1}\G3\G8$, and
$\G1\G1^2=\G4^2=\G0$, with $\G1\G1x=x\G1\G1$ and $\G4x=x\G4$.

\begin{displaymath}
\begin{array}{cccccccccccccccc}
\G0  &\G1  &\G2  &\G3  &\G4  &\G5  &\G6  &\G7  &\G8  &\G9 &\G1\G0 &\G1\G1 &\G1\G2 &\G1\G3 &\G1\G4 &\G1\G5 \\
\G1  &\G0  &\G3  &\G2  &\G5  &\G4  &\G7  &\G6  &\G9  &\G8 &\G1\G1 &\G1\G0 &\G1\G3 &\G1\G2 &\G1\G5 &\G1\G4\\
\G2  &\G3  &\G0  &\G1  &\G6  &\G7  &\G4  &\G5 &\G1\G4 &\G1\G5 &\G1\G2 &\G1\G3 &\G1\G0 &\G1\G1  &\G8  &\G9\\
\G3  &\G2  &\G1  &\G0  &\G7  &\G6  &\G5  &\G4 &\G1\G5 &\G1\G4 &\G1\G3 &\G1\G2 &\G1\G1 &\G1\G0  &\G9  &\G8\\
\G4  &\G5  &\G6  &\G7  &\G1  &\G0  &\G3  &\G2 &\G1\G1 &\G1\G0  &\G8  &\G9 &\G1\G4 &\G1\G5 &\G1\G3 &\G1\G2\\
\G5  &\G4  &\G7  &\G6  &\G0  &\G1  &\G2  &\G3 &\G1\G0 &\G1\G1  &\G9  &\G8 &\G1\G5 &\G1\G4 &\G1\G2 &\G1\G3\\
\G6  &\G7  &\G4  &\G5  &\G3  &\G2  &\G1  &\G0 &\G1\G3 &\G1\G2 &\G1\G4 &\G1\G5  &\G8  &\G9 &\G1\G1 &\G1\G0\\
\G7  &\G6  &\G5  &\G4  &\G2  &\G3  &\G0  &\G1 &\G1\G2 &\G1\G3 &\G1\G5 &\G1\G4  &\G9  &\G8 &\G1\G0 &\G1\G1\\
\G8  &\G9 &\G1\G4 &\G1\G5 &\G1\G0 &\G1\G1 &\G1\G2 &\G1\G3  &\G0  &\G1  &\G4  &\G5  &\G6  &\G7  &\G2  &\G3\\
\G9  &\G8 &\G1\G5 &\G1\G4 &\G1\G1 &\G1\G0 &\G1\G3 &\G1\G2  &\G1  &\G0  &\G5  &\G4  &\G7  &\G6  &\G3  &\G2\\
\G1\G0 &\G1\G1 &\G1\G2 &\G1\G3  &\G9  &\G8 &\G1\G5 &\G1\G4  &\G5  &\G4  &\G0  &\G1  &\G2  &\G3  &\G7  &\G6\\
\G1\G1 &\G1\G0 &\G1\G3 &\G1\G2  &\G8  &\G9 &\G1\G4 &\G1\G5  &\G4  &\G5  &\G1  &\G0  &\G3  &\G2  &\G6  &\G7\\
\G1\G2 &\G1\G3 &\G1\G0 &\G1\G1 &\G1\G5 &\G1\G4  &\G9  &\G8  &\G7  &\G6  &\G2  &\G3  &\G0  &\G1  &\G5  &\G4\\
\G1\G3 &\G1\G2 &\G1\G1 &\G1\G0 &\G1\G4 &\G1\G5  &\G8  &\G9  &\G6  &\G7  &\G3  &\G2  &\G1  &\G0  &\G4  &\G5\\
\G1\G4 &\G1\G5  &\G8  &\G9 &\G1\G2 &\G1\G3 &\G1\G0 &\G1\G1  &\G2  &\G3  &\G6  &\G7  &\G4  &\G5  &\G0  &\G1\\
\G1\G5 &\G1\G4  &\G9  &\G8 &\G1\G3 &\G1\G2 &\G1\G1 &\G1\G0  &\G3  &\G2  &\G7  &\G6  &\G5  &\G4  &\G1  &\G0
\end{array}
\end{displaymath}

 The $\delta V$ for this has 6 variables: $\alpha$, $\beta$, $\gamma$,
$\epsilon$, $\mu$, and $\sigma$; which multiply arrays I denote by
A, B, C, D, E, and F respectively.


\begin{displaymath}
\begin{array}{cccccccccccccccc}
0 &0 &0 &0 &0 &0 &0 &0 &0 &0 &0 &0 &0 &0 &0 &0 \\
0 &0 &0 &0 &0 &0 &0 &0 &0 &0 &0 &0 &0 &0 &0 &0 \\
0 &0 &0 &0 &0 &0 &0 &0 &0 &0 &0 &0 &0 &0 &0 &0 \\
0 &0 &0 &0 &0 &0 &0 &0 &0 &0 &0 &0 &0 &0 &0 &0 \\
0 &0 &0 &0 &0 &0 &0 &0 & \gamma &-\gamma &\mu &-\mu &\sigma &-\sigma &\epsilon &-\epsilon \\
0 &0 &0 &0 &0 &0 &0 &0 & -\gamma &\gamma &-\mu &\mu &-\sigma &\sigma &-\epsilon &\epsilon \\
0 &0 &0 &0 &0 &0 &0 &0 &\epsilon &-\epsilon &\sigma &-\sigma &\mu &-\mu &\gamma &-\gamma \\
0 &0 &0 &0 &0 &0 &0 &0 &-\epsilon &\epsilon &-\sigma &\sigma &-\mu &\mu &-\gamma &\gamma \\
0 &0 &0 &0 &\gamma &-\gamma &\epsilon &-\epsilon &0 &0 &\alpha &-\alpha &\beta &-\beta &0 &0 \\
0 &0 &0 &0 &-\gamma &\gamma &-\epsilon &\epsilon &0 &0 &-\alpha &\alpha &-\beta &\beta &0 &0 \\
0 &0 &0 &0 &\mu &-\mu &\sigma &-\sigma &-\alpha &\alpha &0 &0 &0 &0 &-\beta &\beta \\
0 &0 &0 &0 &-\mu &\mu &-\sigma &\sigma &\alpha &-\alpha &0 &0 &0 &0 &\beta &-\beta \\
0 &0 &0 &0 &\sigma &-\sigma &\mu &-\mu &-\beta &\beta &0 &0 &0 &0 &-\alpha &\alpha \\
0 &0 &0 &0 &-\sigma &\sigma &-\mu &\mu &\beta &-\beta &0 &0 &0 &0 &\alpha &-\alpha \\
0 &0 &0 &0 &\epsilon &-\epsilon &\gamma &-\gamma &0 &0 &\beta &-\beta &\alpha &-\alpha &0 &0\\
0 &0 &0 &0 &-\epsilon &\epsilon &-\gamma &\gamma &0 &0 &-\beta &\beta &-\alpha &\alpha &0 &0
\end{array}
\end{displaymath}

The eigenvalues are given by the solutions to
\begin{displaymath}
0 = X^{12} \left(
\begin{array}{c}
X^4  \\
+8 X^2 \left( \alpha^2 \right.+ \beta^2 -\epsilon^2 -\gamma^2 -\mu^2 \left. -\sigma^2 \right)\\
+ 32 \left(
\begin{array}{ccccc}
 \epsilon^2 \mu^2 &+ \epsilon^2\sigma^2 &+\gamma^2\mu^2 &+\gamma^2 \sigma^2
&\alpha^2 \beta^2 \\
 -\alpha^2\epsilon^2 &-\alpha^2\gamma^2 &-\alpha^2 \mu^2 &-\alpha^2 \sigma^2 
&-\beta^2\epsilon^2 \\
 -\beta^2 \gamma^2 &-\beta^2 \mu^2 &-\beta^2 \sigma^2 &-\mu^2 \sigma^2
&-\epsilon^2 \gamma^2
\end{array} \right) \\
 +128\alpha\beta\epsilon\gamma + 128\alpha\beta\mu\sigma
 -128\epsilon\gamma\mu\sigma    \\
 +16 \alpha^4 +16\beta^4 + 16\epsilon^4 + 16\gamma^4 + 16\mu^4 +16\sigma^4 
\end{array}
\right)
\end{displaymath}

 The generator commutation relations are as follows:

\begin{eqnarray*}
\left[ A,B \right] = 0 \quad\left[ C,D \right] = 0 \quad\left[ E,F \right] = 0\\
\left[ C,E \right] = A \quad\left[ E,A \right] = -C \quad\left[ A,C \right] = -E\\
\left[ C,F \right] = B \quad\left[ F,B \right] = -C \quad\left[ B,C \right] = -F\\
\left[ D,E \right] = B \quad\left[ E,B \right] = -D \quad\left[ B,D \right] = -E\\
\left[ D,F \right] = A \quad\left[ F,A \right] = -D \quad\left[ A,D \right] = -F
\end{eqnarray*}
\subsection{16-element group 5}
% 16_5.DAT
 Another 16-element group is generated by
$\G1^2 = \G0$, $\G2^2 = \G4$, $\G4^2=\G8^2=\G0$, $\G8=\G1^{-1}\G2^{-1}\G1\G2$,
$\G8x = x\G8$, and  $\G4x = x\G4$.  The product (Cayley) table here is
\begin{displaymath}
\begin{array}{cccccccccccccccc}
\G0&  \G1&  \G2&  \G3&  \G4&  \G5&  \G6&  \G7&  \G8&  \G9& \G1\G0& \G1\G1& \G1\G2& \G1\G3& \G1\G4& \G1\G5 \\
\G1&  \G0&  \G3&  \G2&  \G5&  \G4&  \G7&  \G6&  \G9&  \G8& \G1\G1& \G1\G0& \G1\G3& \G1\G2& \G1\G5& \G1\G4 \\
\G2& \G1\G1&  \G4& \G1\G3&  \G6& \G1\G5&  \G0&  \G9& \G1\G0&  \G3& \G1\G2&  \G5& \G1\G4&  \G7&  \G8&  \G1 \\
\G3& \G1\G0&  \G5& \G1\G2&  \G7& \G1\G4&  \G1&  \G8& \G1\G1&  \G2& \G1\G3& \G4& \G1\G5&  \G6&  \G9&  \G0 \\
\G4&  \G5&  \G6&  \G7&  \G0&  \G1&  \G2&  \G3& \G1\G2& \G1\G3& \G1\G4& \G1\G5&  \G8&  \G9& \G1\G0& \G1\G1 \\
\G5&  \G4&  \G7&  \G6&  \G1&  \G0&  \G3&  \G2& \G1\G3& \G1\G2& \G1\G5& \G1\G4&  \G9&  \G8& \G1\G1& \G1\G0 \\
\G6& \G1\G5&  \G0&  \G9&  \G2& \G1\G1&  \G4& \G1\G3& \G1\G4&  \G7&  \G8&  \G1& \G1\G0&  \G3& \G1\G2&  \G5 \\
\G7& \G1\G4&  \G1&  \G8&  \G3& \G1\G0&  \G5& \G1\G2& \G1\G5&  \G6&  \G9&  \G0& \G1\G1&  \G2& \G1\G3&  \G4 \\
\G8&  \G9& \G1\G0& \G1\G1& \G1\G2& \G1\G3& \G1\G4& \G1\G5&  \G0&  \G1&  \G2&  \G3&  \G4&  \G5&  \G6&  \G7 \\
\G9&  \G8& \G1\G1& \G1\G0& \G1\G3& \G1\G2& \G1\G5& \G1\G4&  \G1&  \G0&  \G3&  \G2&  \G5&  \G4&  \G7&  \G6 \\
\G1\G0&  \G3& \G1\G2&  \G5& \G1\G4&  \G7&  \G8&  \G1&  \G2& \G1\G1&  \G4& \G1\G3&  \G6& \G1\G5&  \G0&  \G9 \\
\G1\G1&  \G2& \G1\G3&  \G4& \G1\G5&  \G6&  \G9&  \G0&  \G3& \G1\G0&  \G5& \G1\G2&  \G7& \G1\G4&  \G1&  \G8 \\
\G1\G2& \G1\G3& \G1\G4& \G1\G5&  \G8&  \G9& \G1\G0& \G1\G1&  \G4&  \G5&  \G6&  \G7&  \G0&  \G1&  \G2&  \G3 \\
\G1\G3& \G1\G2& \G1\G5& \G1\G4&  \G9&  \G8& \G1\G1& \G1\G0&  \G5&  \G4&  \G7&  \G6&  \G1&  \G0&  \G3&  \G2 \\
\G1\G4&  \G7&  \G8&  \G1& \G1\G0&  \G3& \G1\G2&  \G5&  \G6& \G1\G5&  \G0&  \G9&  \G2& \G1\G1&  \G4& \G1\G3\\
\G1\G5&  \G6&  \G9&  \G0& \G1\G1&  \G2& \G1\G3&  \G4&  \G7& \G1\G4&  \G1&  \G8&  \G3& \G1\G0&  \G5& \G1\G2
\end{array}
\end{displaymath}

 This results in a $\delta V$ of the form
\begin{displaymath}
\begin{array}{cccccccccccccccc}
0 & 0 & 0 & 0 & 0 & 0 & 0 & 0 & 0 & 0 & 0 & 0 & 0 & 0 & 0 & 0 \\
0 & 0 & \mu &-\beta & 0 & 0 &-\sigma &-\epsilon & 0 & 0 &-\mu & \beta & 0 & 0 & \sigma & \epsilon \\
0 & \sigma & 0 &-\alpha & 0 &-\mu & 0 &-\gamma & 0 &-\sigma & 0 & \alpha & 0 & \mu & 0 & \gamma \\
0 &-\epsilon &-\alpha & 0 & 0 &-\beta &-\gamma & 0 & 0 & \epsilon & \alpha & 0 & 0 & \beta & \gamma & 0 \\
0 & 0 & 0 & 0 & 0 & 0 & 0 & 0 & 0 & 0 & 0 & 0 & 0 & 0 & 0 & 0 \\
0 & 0 &-\sigma &-\epsilon & 0 & 0 & \mu &-\beta & 0 & 0 & \sigma & \epsilon & 0 & 0 &-\mu & \beta \\
0 &-\mu & 0 &-\gamma & 0 & \sigma & 0 &-\alpha & 0 & \mu & 0 & \gamma & 0 &-\sigma & 0 & \alpha \\
0 &-\beta &-\gamma & 0 & 0 &-\epsilon &-\alpha & 0 & 0 & \beta & \gamma & 0 & 0 & \epsilon & \alpha & 0 \\
0 & 0 & 0 & 0 & 0 & 0 & 0 & 0 & 0 & 0 & 0 & 0 & 0 & 0 & 0 & 0 \\
0 & 0 &-\mu & \beta & 0 & 0 & \sigma & \epsilon & 0 & 0 & \mu &-\beta & 0 & 0 &-\sigma &-\epsilon \\
0 &-\sigma & 0 & \alpha & 0 & \mu & 0 & \gamma & 0 & \sigma & 0 &-\alpha & 0 &-\mu & 0 &-\gamma \\
0 & \epsilon & \alpha & 0 & 0 & \beta & \gamma & 0 & 0 &-\epsilon &-\alpha & 0 & 0 &-\beta &-\gamma & 0 \\
0 & 0 & 0 & 0 & 0 & 0 & 0 & 0 & 0 & 0 & 0 & 0 & 0 & 0 & 0 & 0 \\
0 & 0 & \sigma & \epsilon & 0 & 0 &-\mu & \beta & 0 & 0 &-\sigma &-\epsilon & 0 & 0 & \mu &-\beta \\
0 & \mu & 0 & \gamma & 0 &-\sigma & 0 & \alpha & 0 &-\mu & 0 &-\gamma & 0 & \sigma & 0 &-\alpha \\
0 & \beta & \gamma & 0 & 0 & \epsilon & \alpha & 0 & 0 &-\beta &-\gamma & 0 & 0 &-\epsilon &-\alpha & 0
\end{array}
\end{displaymath}

 The commutation relations among the generators are as follows,
where $2\alpha A + 2\beta B + 2\gamma C + 2\epsilon D + 2\mu E + 2\sigma F =
\delta V$:
\begin{displaymath}
\begin{array}{ccccc}
 \left[ A,C\right] = 0 & \left[ B,D \right] = A &  \left[ B,F \right] = -C &
 \left[ E,D\right] = C  &  \left[ E,F\right] = -A \\
 \left[ B,E\right] = 0 &  \left[ D,A\right] = E & \left[ F,C \right] = -E &
 \left[ D,C\right] = B &  \left[ F,A \right] = -B \\
 \left[ D,F\right] = 0 &  \left[ A,E \right] = -D &  \left[ C,E\right] = F &
 \left[ C,B\right] = -D & \left[ A,B\right] = F
\end{array}
\end{displaymath}
 The eigenfunction for the differential array is given below.  There are
10 constant eigenvectors with eigenvalue 0 (for the differential array)
or 1 (for the full transformation.
\begin{displaymath}
0 = X^{12} \left( 
\begin{array}{c}
X^4 \\ 
+X^2  \left(-16\sigma\mu + \left(-16\epsilon\beta + 
\left(-8\alpha^2 - 8\gamma^2\right)\right)\right) \\ 
+ \\
-16\mu^4 + \left(32\beta^2 + \left(32\sigma^2 + 
\left(32\epsilon^2 + 64\gamma\alpha\right)\right)\right)\mu^2 \\
+ \left(128\epsilon\sigma\beta + 
\left(64\alpha^2 + 64\gamma^2\right)\sigma\right)\mu \\
+ \left(-16\beta^4 + \left(32\sigma^2 + 
\left(32\epsilon^2 - 64\gamma\alpha\right)\right)\beta^2 \right. \\
+ \left(64\alpha^2 + 64\gamma^2\right)\epsilon\beta + \left(-16\sigma^4 + 
\left(32\epsilon^2 + 64\gamma\alpha\right)\sigma^2 \right. \\
+ \left. \left. \left(-16\epsilon^4 - 64\gamma\alpha\epsilon^2 +
\left(16\alpha^4 - 32\gamma^2\alpha^2 + 
16\gamma^4\right)\right)\right)\right)
\end{array} \right)
\end{displaymath}
\subsection{16-element group 6}
% 16_6.dat
There is a 6'th 16-element non-abelian group, generated by the following
relations:  $\G1^2=\G4$, $\G2^2=\G8$, $\G4=\G1^{-1}\G2^{-1}\G1 \G2$,
and $\G4^2=\G8^2=\G0$, with $\G4x=x\G4$, $\G8x=x\G8$.  It's product (Cayley) table
is given by
\begin{displaymath}
\begin{array}{cccccccccccccccc}
 \G0&  \G1&  \G2&  \G3&  \G4&  \G5&  \G6&  \G7&  \G8&  \G9& \G1\G0& \G1\G1& \G1\G2& \G1\G3& \G1\G4& \G1\G5 \\
 \G1&  \G4&  \G3&  \G6&  \G5&  \G0&  \G7&  \G2&  \G9& \G1\G2& \G1\G1& \G1\G4& \G1\G3&  \G8& \G1\G5& \G1\G0 \\
 \G2&  \G7&  \G8& \G1\G3&  \G6&  \G3& \G1\G2&  \G9& \G1\G0& \G1\G5&  \G0&  \G5& \G1\G4& \G1\G1&  \G4&  \G1 \\
 \G3&  \G2&  \G9&  \G8&  \G7&  \G6& \G1\G3& \G1\G2& \G1\G1& \G1\G0&  \G1&  \G0& \G1\G5& \G1\G4&  \G5&  \G4 \\
 \G4&  \G5&  \G6&  \G7&  \G0&  \G1&  \G2&  \G3& \G1\G2& \G1\G3& \G1\G4& \G1\G5&  \G8&  \G9& \G1\G0& \G1\G1 \\
 \G5&  \G0&  \G7&  \G2&  \G1&  \G4&  \G3&  \G6& \G1\G3&  \G8& \G1\G5& \G1\G0&  \G9& \G1\G2& \G1\G1& \G1\G4 \\
 \G6&  \G3& \G1\G2&  \G9&  \G2&  \G7&  \G8& \G1\G3& \G1\G4& \G1\G1&  \G4&  \G1& \G1\G0& \G1\G5&  \G0&  \G5 \\
 \G7&  \G6& \G1\G3& \G1\G2&  \G3&  \G2&  \G9&  \G8& \G1\G5& \G1\G4&  \G5&  \G4& \G1\G1& \G1\G0&  \G1&  \G0 \\
 \G8&  \G9& \G1\G0& \G1\G1& \G1\G2& \G1\G3& \G1\G4& \G1\G5&  \G0&  \G1&  \G2&  \G3&  \G4&  \G5&  \G6&  \G7 \\
 \G9& \G1\G2& \G1\G1& \G1\G4& \G1\G3&  \G8& \G1\G5& \G1\G0&  \G1&  \G4&  \G3&  \G6&  \G5&  \G0&  \G7&  \G2 \\
\G1\G0& \G1\G5&  \G0&  \G5& \G1\G4& \G1\G1&  \G4&  \G1&  \G2&  \G7&  \G8& \G1\G3&  \G6&  \G3& \G1\G2&  \G9 \\
\G1\G1& \G1\G0&  \G1&  \G0& \G1\G5& \G1\G4&  \G5&  \G4&  \G3&  \G2&  \G9&  \G8&  \G7&  \G6& \G1\G3& \G1\G2 \\
\G1\G2& \G1\G3& \G1\G4& \G1\G5&  \G8&  \G9& \G1\G0& \G1\G1&  \G4&  \G5&  \G6&  \G7&  \G0&  \G1&  \G2&  \G3 \\
\G1\G3&  \G8& \G1\G5& \G1\G0&  \G9& \G1\G2& \G1\G1& \G1\G4&  \G5&  \G0&  \G7&  \G2&  \G1&  \G4&  \G3&  \G6 \\
\G1\G4& \G1\G1&  \G4&  \G1& \G1\G0& \G1\G5&  \G0&  \G5&  \G6&  \G3& \G1\G2&  \G9&  \G2&  \G7&  \G8& \G1\G3 \\
\G1\G5& \G1\G4&  \G5&  \G4& \G1\G1& \G1\G0&  \G1&  \G0&  \G7&  \G6& \G1\G3& \G1\G2&  \G3&  \G2&  \G9&  \G8 \\
\end{array}
\end{displaymath}

Solving for the differential gives the following somewhat unusual array.
Note that an element is no longer equal to the absolute value of the element
in its transpose position.
\begin{displaymath}
\begin{array}{cccccccccccccccc}
0 & 0 & 0 & 0 & 0 & 0 & 0 & 0 &  0 & 0 & 0 & 0 & 0 & 0 & 0 & 0 \\
0 & 0 & \mu &-\beta & 0 & 0 &-\mu & \beta &  0 & 0 &-\sigma &-\epsilon & 0 & 0 & \sigma & \epsilon \\
0 &-\sigma & 0 &-\alpha & 0 & \sigma & 0 & \alpha &  0 & \mu & 0 &-\gamma & 0 &-\mu & 0 & \gamma \\
0 &-\epsilon & \alpha & 0 & 0 & \epsilon &-\alpha & 0 &  0 &-\beta & \gamma & 0 & 0 & \beta &-\gamma & 0 \\
0 & 0 & 0 & 0 & 0 & 0 & 0 & 0 &  0 & 0 & 0 & 0 & 0 & 0 & 0 & 0 \\
0 & 0 &-\mu & \beta & 0 & 0 & \mu &-\beta &  0 & 0 & \sigma & \epsilon & 0 & 0 &-\sigma &-\epsilon \\
0 & \sigma & 0 & \alpha & 0 &-\sigma & 0 &-\alpha &  0 &-\mu & 0 & \gamma & 0 & \mu & 0 &-\gamma \\
0 & \epsilon &-\alpha & 0 & 0 &-\epsilon & \alpha & 0 &  0 & \beta &-\gamma & 0 & 0 &-\beta & \gamma & 0 \\
0 & 0 & 0 & 0 & 0 & 0 & 0 & 0 &  0 & 0 & 0 & 0 & 0 & 0 & 0 & 0 \\
0 & 0 &-\sigma &-\epsilon & 0 & 0 & \sigma & \epsilon &  0 & 0 & \mu &-\beta & 0 & 0 &-\mu & \beta \\
0 & \mu & 0 &-\gamma & 0 &-\mu & 0 & \gamma &  0 &-\sigma & 0 &-\alpha & 0 & \sigma & 0 & \alpha \\
0 &-\beta & \gamma & 0 & 0 & \beta &-\gamma & 0 &  0 &-\epsilon & \alpha & 0 & 0 & \epsilon &-\alpha & 0 \\
0 & 0 & 0 & 0 & 0 & 0 & 0 & 0 &  0 & 0 & 0 & 0 & 0 & 0 & 0 & 0 \\
0 & 0 & \sigma & \epsilon & 0 & 0 &-\sigma &-\epsilon &  0 & 0 &-\mu & \beta & 0 & 0 & \mu &-\beta \\
0 &-\mu & 0 & \gamma & 0 & \mu & 0 &-\gamma &  0 & \sigma & 0 & \alpha & 0 &-\sigma & 0 &-\alpha \\
0 & \beta &-\gamma & 0 & 0 &-\beta & \gamma & 0 &  0 & \epsilon &-\alpha & 0 & 0 &-\epsilon & \alpha & 0 
\end{array}
\end{displaymath}
 The eigenvalue of this differential transform are (found by REDUCE)
\begin{displaymath}
X^{12} \left(
\begin{array}{ccccc}
16\alpha^4 & -16\sigma^4 & -16\mu^4 & +16\gamma^4 & -16\beta^4 \\
-16\epsilon^4 & -64\alpha^2\beta\epsilon  & +32\mu^2\sigma^2 &
-32\alpha^2\gamma^2 & +64\alpha^2\mu\sigma \\
+64\alpha\beta^2\gamma & +64\alpha\epsilon^2\gamma & +64\alpha\gamma\mu^2 &
+64\alpha\gamma\sigma^2 & +32\beta^2\epsilon^2 \\
-32\beta^2\mu^2 & -32\beta^2\sigma^2 & -64\beta\epsilon\gamma^2 &
-128\beta\epsilon\mu\sigma & -32\epsilon^2\mu^2 \\
-32\epsilon^2\sigma^2 & +64\gamma^2\mu\sigma & & & \\
-16\beta\epsilon X^2 & +8\gamma^2 X^2 & +16\mu\sigma X^2 & +8\alpha^2 X^2 & \\
+X^4 & & & &
\end{array}
\right)
\end{displaymath}
 Natural generators are given by $\delta V = 2\alpha A + 2\beta B +
2\gamma C + 2\epsilon D + 2\mu E + 2\sigma F$.
The commutation relation among these generators are
\begin{displaymath}
\begin{array}{ccccc}
\left[A,C\right]=0 & \left[B,E\right]=A & \left[B,F\right]=-C &
\left[D,E\right]=-C & \left[D,F\right]=A \\
\left[B,D\right]=0 & \left[E,A\right]=B & \left[F,C\right]=-B &
\left[E,C\right]=-D & \left[F,A\right]=D \\
\left[E,F\right]=0 & \left[A,B\right]=-F & \left[C,B\right]=E &
\left[C,D\right]=F & \left[A,D\right]=-E
\end{array}
\end{displaymath}
\subsection{16-element group 7}
% 16_7
The seventh of our non-abelian 16-element groups is generated by the
relations $\G4^2 = \G8^2 = \G0$, $\G1^2=\G2$, $\G2^2=\G4$, 
and $\G4=\G8^{-1}\G1^{-1}\G8\G1$.
\begin{displaymath}
\begin{array}{cccccccccccccccc}
\G0&  \G1&  \G2&  \G3&  \G4&  \G5&  \G6&  \G7&  \G8&  \G9& \G1\G0& \G1\G1& \G1\G2& \G1\G3& \G1\G4& \G1\G5 \\
\G1&  \G2&  \G3&  \G4&  \G5&  \G6&  \G7&  \G0&  \G9& \G1\G0& \G1\G1& \G1\G2& \G1\G3& \G1\G4& \G1\G5&  \G8 \\
\G2&  \G3&  \G4&  \G5&  \G6&  \G7&  \G0&  \G1& \G1\G0& \G1\G1& \G1\G2& \G1\G3& \G1\G4& \G1\G5&  \G8&  \G9 \\
\G3&  \G4&  \G5&  \G6&  \G7&  \G0&  \G1&  \G2& \G1\G1& \G1\G2& \G1\G3& \G1\G4& \G1\G5&  \G8&  \G9& \G1\G0 \\
\G4&  \G5&  \G6&  \G7&  \G0&  \G1&  \G2&  \G3& \G1\G2& \G1\G3& \G1\G4& \G1\G5&  \G8&  \G9& \G1\G0& \G1\G1 \\
\G5&  \G6&  \G7&  \G0&  \G1&  \G2&  \G3&  \G4& \G1\G3& \G1\G4& \G1\G5&  \G8&  \G9& \G1\G0& \G1\G1& \G1\G2 \\
\G6&  \G7&  \G0&  \G1&  \G2&  \G3&  \G4&  \G5& \G1\G4& \G1\G5&  \G8&  \G9& \G1\G0& \G1\G1& \G1\G2& \G1\G3 \\
\G7&  \G0&  \G1&  \G2&  \G3&  \G4&  \G5&  \G6& \G1\G5&  \G8&  \G9& \G1\G0& \G1\G1& \G1\G2& \G1\G3& \G1\G4 \\
\G8& \G1\G3& \G1\G0& \G1\G5& \G1\G2&  \G9& \G1\G4& \G1\G1&  \G0&  \G5&  \G2&  \G7&  \G4&  \G1&  \G6&  \G3 \\
\G9& \G1\G4& \G1\G1&  \G8& \G1\G3& \G1\G0& \G1\G5& \G1\G2&  \G1&  \G6&  \G3&  \G0&  \G5&  \G2&  \G7&  \G4 \\
\G1\G0& \G1\G5& \G1\G2&  \G9& \G1\G4& \G1\G1&  \G8& \G1\G3&  \G2&  \G7&  \G4&  \G1&  \G6&  \G3&  \G0&  \G5 \\
\G1\G1&  \G8& \G1\G3& \G1\G0& \G1\G5& \G1\G2&  \G9& \G1\G4&  \G3&  \G0&  \G5&  \G2&  \G7&  \G4&  \G1&  \G6 \\
\G1\G2&  \G9& \G1\G4& \G1\G1&  \G8& \G1\G3& \G1\G0& \G1\G5&  \G4&  \G1&  \G6&  \G3&  \G0&  \G5&  \G2&  \G7 \\
\G1\G3& \G1\G0& \G1\G5& \G1\G2&  \G9& \G1\G4& \G1\G1&  \G8&  \G5&  \G2&  \G7&  \G4&  \G1&  \G6&  \G3&  \G0 \\
\G1\G4& \G1\G1&  \G8& \G1\G3& \G1\G0& \G1\G5& \G1\G2&  \G9&  \G6&  \G3&  \G0&  \G5&  \G2&  \G7&  \G4&  \G1 \\
\G1\G5& \G1\G2&  \G9& \G1\G4& \G1\G1&  \G8& \G1\G3& \G1\G0&  \G7&  \G4&  \G1&  \G6&  \G3&  \G0&  \G5&  \G2
\end{array}
\end{displaymath}

The resulting $\delta V$ is given by
\begin{displaymath}
\begin{array}{cccccccccccccccc}
0 &0 &0 &0 &0 &0 &0 &0 &0 &0 &0 &0 &0 &0 &0 &0 \\
0 &0 &0 &0 &0 &0 &0 &0 &\mu &-\gamma &-\sigma &-\epsilon &-\mu &\gamma &\sigma &\epsilon \\
0 &0 &0 &0 &0 &0 &0 &0 &0 &0 &0 &0 &0 &0 &0 &0 \\
0 &0 &0 &0 &0 &0 &0 &0 &\sigma &\epsilon &\mu &-\gamma &-\sigma &-\epsilon &-\mu &\gamma \\
0 &0 &0 &0 &0 &0 &0 &0 &0 &0 &0 &0 &0 &0 &0 &0 \\
0 &0 &0 &0 &0 &0 &0 &0 &-\mu &\gamma &\sigma &\epsilon &\mu &-\gamma &-\sigma &-\epsilon \\
0 &0 &0 &0 &0 &0 &0 &0 &0 &0 &0 &0 &0 &0 &0 &0 \\
0 &0 &0 &0 &0 &0 &0 &0 &-\sigma &-\epsilon &-\mu &\gamma &\sigma &\epsilon &\mu &-\gamma \\
0 &\sigma &0 &\mu &0 &-\sigma &0 &-\mu &0 &-\alpha &0 &-\beta &0 &\alpha &0 &\beta \\
0 &-\gamma &0 &-\epsilon &0 &\gamma &0 &\epsilon &\beta &0 &-\alpha &0 &-\beta &0 &\alpha &0 \\
0 &-\mu &0 &\sigma &0 &\mu &0 &-\sigma &0 &\beta &0 &-\alpha &0 &-\beta &0 &\alpha \\
0 &\epsilon &0 &-\gamma &0 &-\epsilon &0 &\gamma &\alpha &0 &\beta &0 &-\alpha &0 &-\beta &0 \\
0 &-\sigma &0 &-\mu &0 &\sigma &0 &\mu &0 &\alpha &0 &\beta &0 &-\alpha &0 &-\beta \\
0 &\gamma &0 &\epsilon &0 &-\gamma &0 &-\epsilon &-\beta &0 &\alpha &0 &\beta &0 &-\alpha &0 \\
0 &\mu &0 &-\sigma &0 &-\mu &0 &\sigma &0 &-\beta &0 &\alpha &0 &\beta &0 &-\alpha \\
0 &-\epsilon &0 &\gamma &0 &\epsilon &0 &-\gamma &-\alpha &0 &-\beta &0 &\alpha &0 &\beta &0
\end{array}
\end{displaymath}
 Once again, the generators themselves are sometimes neither symmetric nor
anti-symmetric.
The natural generators are just the arrays corresponding to the
above differentials, divided by 2.  $\delta V = 2 \alpha A +
2\beta B + 2\gamma C + 2\epsilon D + 2\mu E + 2\sigma F$.
Their commutators are
\begin{displaymath}
\begin{array}{ccccc}
\left[A,B\right]=0 & \left[C,D\right]=A & \left[E,D\right]=B &
\left[C,F\right]=B & \left[E,F\right]=-A \\
\left[C,E\right]=0 & \left[D,A\right]=E & \left[D,B\right]=-C &
\left[F,B\right]=-E & \left[F,A\right]=-C \\
\left[D,F\right]=0 & \left[A,E\right]=-F & \left[B,C\right]=-F &
\left[B,E\right]=D & \left[A,C\right]=-D
\end{array}
\end{displaymath}
 The eigenvalues of this differential transform are the solutions to
the following equation (found by REDUCE):
\begin{displaymath}
X^{12} \left(
\begin{array}{ccccc}
16\alpha^4 & +32\alpha^2\beta^2 & +64\alpha^2\epsilon\gamma &
+32\alpha^2\mu^2 & -32\alpha^2\sigma^2 \\
+16\sigma^4 & +64\alpha\beta\epsilon^2 & -64\alpha\beta\gamma^2 &
-128\alpha\beta\mu\sigma & +16\beta^4 \\
-64\beta^2\epsilon\gamma & -32\beta^2\mu^2 & +32\beta^2\sigma^2 &
+16\epsilon^4 & +32\epsilon^2\gamma^2 \\
-64\epsilon^2\mu\sigma & +64\epsilon\gamma\mu^2 & -64\epsilon\gamma\sigma^2 &
+16\gamma^4 & +64\gamma^2\mu\sigma \\
+16\mu^4 & +32\mu^2\sigma^2 & & & \\
-8\gamma^2 X^2 & -16\mu\sigma X^2 & +16\alpha\beta X^2 & +8\epsilon^2 X^2 & \\
+X^4 & & & &
\end{array}
\right)
\end{displaymath}
% 16_8
\subsection{16-element group 8}
% 16_8
An 8'th 16-element group is generated by the relations
${\G2}^2 ={\G0}$, ${\G1}^2 = {\G2}$, ${\G8}^2={\G1}^{-1}$, ${\G4}^2= {\G2}$,
${\G2} = {\G1}^{-1}{\G4}^{-1}{\G1}{\G4}$, and
${\G1}={\G8}^{-1}{\G4}^{-1}{\G8}{\G4}$.  The product (Cayley) table resulting is

\begin{displaymath}
\begin{array}{cccccccccccccccc}
\G0&\G1&\G2&\G3&\G4&\G5&\G6&\G7&\G8&\G9&\G1\G0&\G1\G1&\G1\G2&\G1\G3&\G1\G4&\G1\G5\\
\G1&\G2&\G3&\G0&\G6&\G7&\G5&\G4&\G1\G0&\G1\G1&\G9&\G8&\G1\G4&\G1\G5&\G1\G3&\G1\G2\\
\G2&\G3&\G0&\G1&\G5&\G4&\G7&\G6&\G9&\G8&\G1\G1&\G1\G0&\G1\G3&\G1\G2&\G1\G5&\G1\G4\\
\G3&\G0&\G1&\G2&\G7&\G6&\G4&\G5&\G1\G1&\G1\G0&\G8&\G9&\G1\G5&\G1\G4&\G1\G2&\G1\G3\\
\G4&\G7&\G5&\G6&\G2&\G0&\G1&\G3&\G1\G4&\G1\G5&\G1\G2&\G1\G3&\G1\G1&\G1\G0&\G9&\G8\\
\G5&\G6&\G4&\G7&\G0&\G2&\G3&\G1&\G1\G5&\G1\G4&\G1\G3&\G1\G2&\G1\G0&\G1\G1&\G8&\G9\\
\G6&\G4&\G7&\G5&\G3&\G1&\G2&\G0&\G1\G3&\G1\G2&\G1\G4&\G1\G5&\G8&\G9&\G1\G1&\G1\G0\\
\G7&\G5&\G6&\G4&\G1&\G3&\G0&\G2&\G1\G2&\G1\G3&\G1\G5&\G1\G4&\G9&\G8&\G1\G0&\G1\G1\\
\G8&\G1\G0&\G9&\G1\G1&\G1\G2&\G1\G3&\G1\G4&\G1\G5&\G3&\G1&\G0&\G2&\G7&\G6&\G4&\G5\\
\G9&\G1\G1&\G8&\G1\G0&\G1\G3&\G1\G2&\G1\G5&\G1\G4&\G1&\G3&\G2&\G0&\G6&\G7&\G5&\G4\\
\G1\G0&\G9&\G1\G1&\G8&\G1\G4&\G1\G5&\G1\G3&\G1\G2&\G0&\G2&\G1&\G3&\G4&\G5&\G6&\G7\\
\G1\G1&\G8&\G1\G0&\G9&\G1\G5&\G1\G4&\G1\G2&\G1\G3&\G2&\G0&\G3&\G1&\G5&\G4&\G7&\G6\\
\G1\G2&\G1\G5&\G1\G3&\G1\G4&\G9&\G8&\G1\G0&\G1\G1&\G4&\G5&\G7&\G6&\G2&\G0&\G1&\G3\\
\G1\G3&\G1\G4&\G1\G2&\G1\G5&\G8&\G9&\G1\G1&\G1\G0&\G5&\G4&\G6&\G7&\G0&\G2&\G3&\G1\\
\G1\G4&\G1\G2&\G1\G5&\G1\G3&\G1\G1&\G1\G0&\G9&\G8&\G6&\G7&\G4&\G5&\G3&\G1&\G2&\G0\\
\G1\G5&\G1\G3&\G1\G4&\G1\G2&\G1\G0&\G1\G1&\G8&\G9&\G7&\G6&\G5&\G4&\G1&\G3&\G0&\G2
\end{array}
\end{displaymath}
 
This has a solution with 9 parameters.


\begin{displaymath}
\begin{array}{ccccccccc}
 0 & 0 & 0 & 0 &  0 & 0 & 0 & 0 & ... \\
 0 & 0 & 0 & 0 &  -\rho & \rho & -\lambda & \lambda &  ... \\
 0 & 0 & 0 & 0 &  0 & 0 & 0 & 0 & ... \\
 0 & 0 & 0 & 0 &  \rho & -\rho & \lambda & -\lambda & ... \\

 0 & \rho & 0 & -\rho &  0 & 0 & -\alpha & \alpha & ... \\
 0 & -\rho & 0 & \rho &  0 & 0 & \alpha & -\alpha & ... \\
 0 & \lambda & 0 & -\lambda &  \alpha & -\alpha & 0 & 0 & ... \\
 0 & -\lambda & 0 & \lambda &  -\alpha & \alpha & 0 & 0 & ... \\

 0 & 0 & 0 & 0 &  \nu & \nu-\mu-\sigma & \sigma-\nu & \mu-\nu & ... \\
 0 & 0 & 0 & 0 &  \nu-\mu-\sigma & \nu & \mu-\nu & \sigma-\nu & ... \\
 0 & 0 & 0 & 0 &  -\nu & \mu+\sigma-\nu & \nu-\sigma & \nu-\mu & ... \\
 0 & 0 & 0 & 0 &  \mu+\sigma-\nu & -\nu & \nu-\mu & \nu-\sigma & ... \\

 0 & \mu & 0 & -\mu &  \beta & \tau & -\tau & -\beta & ... \\
 0 & -\mu & 0 & \mu &  \tau & \beta & -\beta & -\tau &  ... \\
 0 & \sigma & 0 & -\sigma &  -\beta & -\tau & \beta & \tau & ... \\
 0 & -\sigma & 0 & \sigma &  -\tau & -\beta & \tau & \beta & ... \\
& & & & & & & & \\
& & & & & & & & \\
 ... &  0 & 0 & 0 & 0 &  0 & 0 & 0 & 0 \\
 ... & 0 & 0 & 0 & 0 &  -\mu & \mu & -\sigma & \sigma \\
 ... & 0 & 0 & 0 & 0 &  0 & 0 & 0 & 0 \\
 ... & 0 & 0 & 0 & 0 &  \mu & -\mu & \sigma & -\sigma \\
 ... & \nu-\mu-\sigma & \nu & \mu+\sigma-\nu & -\nu &  -\beta & -\tau & \beta & \tau \\
 ... & \nu & \nu-\mu-\sigma & -\nu & \mu+\sigma-\nu &  -\tau & -\beta & \tau & \beta \\
 ... & \mu-\nu & \sigma-\nu & \nu-\mu & \nu-\sigma &  \tau & \beta & -\beta & -\tau \\
 ... & \sigma-\nu & \mu-\nu & \nu-\sigma & \nu-\mu &  \beta & \tau & -\tau & -\beta \\
 ... & 0 & 0 & 0 & 0 &  \rho-\lambda-\epsilon & -\epsilon & \lambda+\epsilon & -\rho+\epsilon \\
 ... & 0 & 0 & 0 & 0 &  -\epsilon & \rho-\lambda-\epsilon & \epsilon-\rho & \lambda+\epsilon \\
 ... & 0 & 0 & 0 & 0 &  \lambda+\epsilon-\rho & \epsilon & -\lambda-\epsilon & \rho-\epsilon \\
 ... & 0 & 0 & 0 & 0 &  \epsilon & \lambda+\epsilon-\rho & \rho-\epsilon & -\lambda-\epsilon \\
 ... & -\epsilon & \rho-\lambda-\epsilon & \epsilon & \lambda+\epsilon-\rho &  0 & 0 & -\alpha & \alpha \\
 ... & \rho-\lambda-\epsilon & -\epsilon & \lambda+\epsilon-\rho & \epsilon &  0 & 0 & \alpha & -\alpha \\
 ... & \epsilon-\rho & \lambda+\epsilon & \rho-\epsilon & -\lambda-\epsilon &  \alpha & -\alpha & 0 & 0 \\
 ... & \lambda+\epsilon & \epsilon-\rho & -\lambda-\epsilon & \rho-\epsilon &  -\alpha & \alpha & 0 & 0 

\end{array}
\end{displaymath}

\begin{displaymath}
\begin{array}{ccc}
\alpha \rightarrow C_1& \beta \rightarrow C_2& \epsilon \rightarrow C_8\\
\rho \rightarrow C_9& \lambda \rightarrow C_3& \mu \rightarrow C_4\\
\nu \rightarrow C_6& \sigma \rightarrow C_5& \tau \rightarrow C_7
\end{array}
\end{displaymath}

If we define new generators based on the old ones, the commutation
relations are greatly simplified:

\begin{displaymath}
\begin{array}{ccc}
D_1 = 1/4 \left( C_2 + C_7 \right) & D_2 = 1/4 C_6 & D_3 = 1/4 C_8 \\
D_4 = 2 C_3 - C_8 & D_5 = C_1 & D_6 = C_4 + C_5 + C_6 \\
D_7 = C_4 - C_5 & D_8 = C_2 - C_7 & D_9 = C_8 + 2 C_9
\end{array}
\end{displaymath}
\begin{displaymath}
\begin{array}{ccc}
\left[ D_1, D_2 \right] = -D_3 & \left[ D_2 , D_3 \right] = D_1  &  \left[ D_3 , D_1 \right] = D_2 \\
\left[ D_4, D_5 \right] = 2 D_9 & \left[ D_4 , D_6 \right] = 4 D_8  &  \left[ D_4 , D_8 \right] = -4 D_6 \\
\left[ D_4, D_9 \right] = -8 D_5 & \left[ D_5 , D_6 \right] = -2 D_7  &  \left[ D_5 , D_7 \right] = 2 D_6 \\
\left[ D_5, D_9 \right] = 2 D_4 & \left[ D_6 , D_7 \right] = -4 D_5  &  \left[ D_6 , D_8 \right] = 2 D_4 \\
\left[ D_7, D_8 \right] = 2 D_9 & \left[ D_7 , D_9 \right] = -4 D_8  &  \left[ D_8 , D_9 \right] = 4 D_7 
\end{array}
\end{displaymath}

The transformation partitions neatly into a transformation based on
$D_1$, $D_2$, and $D_3$, and a transformation based on the other 6.

The group has four one-dimensional and three two-dimensional representations.
If I can partition the transformation based on 6 parameters into 2 based
on three it may be possible to link the N-dimensional representations with
the transformations.  This does not seem to be possible.  Either
I have an error in the generator commutations or I must abandon the
hypothesis that
the transformations may be partitioned into transformations of the independent
representations.


The Casimir invariant is simple:
$$
-1/32 C_1^2 + 1/32 C_2^2 + 1/32 C_3^2 -1/64 C_4^2 - 1/8 C_5^2 -1/16 C_6^2
-1/32 C_7^2 - 1/16 C_8^2 - 1/32 C_9^2
$$


% 16_9.dat
\subsection{16-element group 9}
 The ninth non-abelian 16-element group is generated by the following
relations: $\G1^2=\G2$, $\G2^2=\G4$, $\G4^2=\G0$, $\G8^2=\G0$,
$\G4=\G2^{-1}\G8^{-1}\G2\G8$, and $\G2=\G1^{-1}\G8^{-1}\G1\G8$.  The
resulting product (Cayley) table is:

\begin{displaymath}
\begin{array}{cccccccccccccccc}
\G0& \G1& \G2& \G3& \G4& \G5& \G6& \G7& \G8& \G9& \G1\G0& \G1\G1& \G1\G2& \G1\G3& \G1\G4& \G1\G5 \\
\G1& \G2& \G3& \G4& \G5& \G6& \G7& \G0& \G9& \G1\G0& \G1\G1& \G1\G2& \G1\G3& \G1\G4& \G1\G5& \G8 \\
\G2& \G3& \G4& \G5& \G6& \G7& \G0& \G1& \G1\G0& \G1\G1& \G1\G2& \G1\G3& \G1\G4& \G1\G5& \G8& \G9 \\
\G3& \G4& \G5& \G6& \G7& \G0& \G1& \G2& \G1\G1& \G1\G2& \G1\G3& \G1\G4& \G1\G5& \G8& \G9& \G1\G0 \\
\G4& \G5& \G6& \G7& \G0& \G1& \G2& \G3& \G1\G2& \G1\G3& \G1\G4& \G1\G5& \G8& \G9& \G1\G0& \G1\G1 \\
\G5& \G6& \G7& \G0& \G1& \G2& \G3& \G4& \G1\G3& \G1\G4& \G1\G5& \G8& \G9& \G1\G0& \G1\G1& \G1\G2 \\
\G6& \G7& \G0& \G1& \G2& \G3& \G4& \G5& \G1\G4& \G1\G5& \G8& \G9& \G1\G0& \G1\G1& \G1\G2& \G1\G3 \\
\G7& \G0& \G1& \G2& \G3& \G4& \G5& \G6& \G1\G5& \G8& \G9& \G1\G0& \G1\G1& \G1\G2& \G1\G3& \G1\G4 \\
\G8& \G1\G1& \G1\G4& \G9& \G1\G2& \G1\G5& \G1\G0& \G1\G3& \G0& \G3& \G6& \G1& \G4& \G7& \G2& \G5 \\
\G9& \G1\G2& \G1\G5& \G1\G0& \G1\G3& \G8& \G1\G1& \G1\G4& \G1& \G4& \G7& \G2& \G5& \G0& \G3& \G6 \\
\G1\G0& \G1\G3& \G8& \G1\G1& \G1\G4& \G9& \G1\G2& \G1\G5& \G2& \G5& \G0& \G3& \G6& \G1& \G4& \G7 \\
\G1\G1& \G1\G4& \G9& \G1\G2& \G1\G5& \G1\G0& \G1\G3& \G8& \G3& \G6& \G1& \G4& \G7& \G2& \G5& \G0 \\
\G1\G2& \G1\G5& \G1\G0& \G1\G3& \G8& \G1\G1& \G1\G4& \G9& \G4& \G7& \G2& \G5& \G0& \G3& \G6& \G1 \\
\G1\G3& \G8& \G1\G1& \G1\G4& \G9& \G1\G2& \G1\G5& \G1\G0& \G5& \G0& \G3& \G6& \G1& \G4& \G7& \G2 \\ 
\G1\G4& \G9& \G1\G2& \G1\G5& \G1\G0& \G1\G3& \G8& \G1\G1& \G6& \G1& \G4& \G7& \G2& \G5& \G0& \G3 \\
\G1\G5& \G1\G0& \G1\G3& \G8& \G1\G1& \G1\G4& \G9& \G1\G2& \G7& \G2& \G5& \G0& \G3& \G6& \G1& \G4
\end{array}
\end{displaymath}
 The differential transform array is given by
\begin{displaymath}
\begin{array}{ccccccccc}
0 & 0 & 0 & 0 & 0 & 0 & 0 & 0 & \\
0 & 0 & 0 & 0 & 0 & 0 & 0 & 0 & \\
0 & 0 & 0 & 0 & 0 & 0 & 0 & 0 & .\\
0 & 0 & 0 & 0 & 0 & 0 & 0 & 0 & \\
0 & 0 & 0 & 0 & 0 & 0 & 0 & 0 & .\\
0 & 0 & 0 & 0 & 0 & 0 & 0 & 0 & \\
0 & 0 & 0 & 0 & 0 & 0 & 0 & 0 & .\\
0 & 0 & 0 & 0 & 0 & 0 & 0 & 0 & \\
0 & \chi & -\mu-\delta &-\chi& 0 & -\sigma&\mu+\delta& \sigma& .\\
0 & \mu &\sigma-\epsilon& -\mu& 0& \nu& \epsilon-\sigma& -\nu& \\
0 & \epsilon &-\delta-\nu&-\epsilon& 0&\sigma-\epsilon-\chi& \nu+\delta& \chi+\epsilon-\sigma& .\\
0 & \delta &\epsilon+\chi& -\delta& 0& -\mu-\nu-\delta& -\epsilon-\chi& \mu+\delta+\nu& \\
0 & -\sigma & \mu+\delta& \sigma& 0& \chi& -\mu-\delta& -\chi& .\\
0 & \nu & \epsilon-\sigma& -\nu& 0& \mu& \sigma-\epsilon& -\mu& \\
0 & \sigma-\epsilon-\chi & \nu+\delta& \chi+\epsilon-\sigma& 0& \epsilon& -\nu-\delta& -\epsilon& .\\
0 & -\mu-\delta-\nu & -\epsilon-\chi& \mu+\delta+\nu& 0& \delta& \epsilon+\chi& -\delta& \\
 & & & & & & & & \\
 & 0 & 0 & 0 & 0 & 0 & 0 & 0 & 0 \\
. & -\sigma & \mu& \sigma-\epsilon-\chi& \delta& \chi & \nu& \epsilon& -\mu-\delta-\nu \\
 & -\mu-\delta&\epsilon-\sigma&-\delta-\nu&-\chi-\epsilon&\mu+\delta&\sigma-\epsilon&\delta+\nu&\epsilon+\chi \\
. &  \sigma &-\mu&-\sigma+\epsilon+\chi&-\delta&-\chi &-\nu& -\epsilon& \mu+\delta+\nu \\
 & 0 & 0 & 0 & 0 & 0 & 0 & 0 & 0 \\
. & \chi & \nu & \epsilon &-\mu-\delta-\nu&-\sigma& \mu&\sigma-\epsilon-\chi& \delta \\
 & \mu+\delta&\sigma-\epsilon&\delta+\nu&\epsilon+\chi &-\mu-\delta&\epsilon-\sigma&-\delta-\nu&-\chi-\epsilon \\
. & -\chi &-\nu &-\epsilon &\mu+\delta+\nu& \sigma&-\mu&-\sigma+\epsilon+\chi&-\delta \\
 & 0& -\alpha& -\beta& \alpha& 0 & -\gamma& \beta& \gamma \\
. & \gamma & 0& -\alpha& -\beta& \alpha& 0& -\gamma& \beta \\
 & \beta& \gamma& 0& -\alpha& -\beta& \alpha& 0& -\gamma \\
. & -\gamma& \beta& \gamma& 0& -\alpha& -\beta& \alpha& 0 \\
 & 0& -\gamma& \beta& \gamma& 0& -\alpha& -\beta& \alpha \\
. & \alpha& 0& -\gamma& \beta& \gamma& 0& -\alpha& -\beta \\
 & -\beta& \alpha& 0& -\gamma& \beta& \gamma& 0& -\alpha \\
 & -\alpha& -\beta& \alpha& 0& -\gamma& \beta& \gamma& 0
\end{array}
\end{displaymath}
 The commutation relations among the generators are given in the
following table.  $\delta V = \alpha A + \beta B + \gamma C +
\delta D + \mu M + \nu N + \sigma S + \epsilon E + \chi Q$.
\begin{displaymath}
\begin{array}{cccc}
\left[S,M\right]=-2A & \left[S,E\right]=-2B & \left[S,Q\right]=-2B & \left[S,D\right]=-2A+2G \\
\left[S,N\right]=-2A & \left[S,A\right]=-M+2D-N & \left[S,B\right]=S+E+Q & \left[S,G\right]=-M-N \\
\left[M,E\right]=-2A+2G & \left[M,Q\right]=-2A & \left[M,D\right]=2B & \left[M,N\right]=2B \\
\left[M,A\right]=S-Q & \left[M,B\right]=-M+D+N & \left[M,G\right]=S+2E-Q & \left[E,Q\right]=2B \\
\left[E,D\right]=2A-2G & \left[E,N\right]=0 & \left[E,A\right]=-M-D+N & \left[E,B\right]=-2S-2Q \\
\left[E,G\right]=M+D-N & \left[Q,D\right]=0 & \left[Q,N\right]=2G & \left[Q,A\right]=-D+2N \\
\left[Q,B\right]=-S+E-Q & \left[Q,G\right]=2M-D & \left[D,N\right]=2B & \left[D,A\right]=S-E+Q \\
\left[D,B\right]=-2M+2N & \left[D,G\right]=-S+E-Q & \left[N,A\right]=2S+E & \left[N,B\right]=-M-D+N \\
\left[N,G\right]=E-2Q & \left[A,B\right]=0 & \left[A,G\right]=0 & \left[B,G\right]=0
\end{array}
\end{displaymath}
\end{document}
