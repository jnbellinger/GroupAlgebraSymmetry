% LaTex document
%\pagelayout{normal}

\documentclass{revtex4-1}
%\usepackage{CJK}

%\documentclass[12pt]{article}
\usepackage{amsmath}


%\def\half{{1 \over 2}}

%\topmargin -0.5in
%\oddsidemargin 0in
%\evensidemargin 0in
%\textheight 8.9in
%\textwidth 6.5in
%\parskip 3pt plus 1pt minus 0.5pt

\begin{document}
%\begin {CJK*} {GB} { } 
\title{ Special transformations over a space defined over a finite group}
\author{James N. Bellinger}
\affiliation{University of Wisconsin at Madison}
\date{\today}

%\end{CJK*}

\begin{abstract}

"Special" transformations in a group algebra map the elements representing the elements of the finite group into a set which preserves the
original finite group operations.

Continuous special transformations of dihedral groups of $>=6$ have a maximum of $3(p-1)/2$ independent parameters if $p$ 
is odd, and $3(p-2)/2$ if $p$ is even.

\par The special transformations $T_H$ of a group $H$ which is a subgroup of $G$ 
will sometimes be a subset of the transformations $T_G$ over $G$,
but not in general.  In particular, if $G = H \otimes Z_2$, $T_H$
is a subset of $T_G$.  Therefore in general, if a subgroup $H$ of $G$
transforms as $T_{GH}$ as part of $T_G$, $T_{GH}$ will not be the same as
the transforms $T_H$ of $H$ as a stand-alone group.
\par Unless the original group objects are physical objects,
these continuous transformations are unlikely to have physical meaning.
\end{abstract}
\maketitle


\section{Introduction}
As described in an earlier paper \footnote{Symmetries of preon interactions modeled as a finite group. 
J.N. Bellinger J.Math.Phys.38:3414-3426,1997}, non-abelian finite groups can sustain
continuous special transformations in a space that maps the elements corresponding to the elements of the finite group into a set which
preserves the original
group structure.  Hereafter when I refer to a transformation it is one of these "special" transformations.
These transformations are of the form of $n \times n$ matrices, where $n$ is the
order of the group.  Name such a transform matrix $V$; the equation governing its differential
changes is $$\delta V_{i,kj^{-1}} + \delta V_{j,i^{-1}k} = \delta V_{ij,k}$$.  The $i$, $j$, and $k$ are the numbered group elements, 
such that (e.g.) $kj^{-1}$ is the number of the element given by the group product of group element $k$ and the inverse of $j$.

\par While it is possible to solve for the generators of such a transformation for any given
group, it is good to have general solutions available.  One simple example is dihedral groups.

\section{Dihedral groups:  $2 \times$ p}
Dihedral groups of order 2p obey $a^2=0$ and $b^p=0$
and $ba=ab^{-1}$ for appropriate selections of $a$ and $b$.  It is convenient
to number the group elements from $\{0\dots 2p-1\}$ mapping $i \rightarrow b^i$ for 
$i \in \{0\dots p-1\}$ and $i \rightarrow ab^{i-p}$ for $i \in \{p\dots 2p-1\}$.
Name the first subset $A$ and the second one $B$.   The subset $A$ is an abelian subgroup,
and we already know that the $\delta V_{i,j}$ components are all $0$ when $i$ and $j$ are
in the abelian subgroup.

\par The $\delta V$ array then neatly partitions into 4 sub-matrices:
\begin{equation}
\delta V = \left(\begin{array}{cc} V1 & V2 \\ V3 & V4
\end{array}\right)
\end{equation}

$$\delta V_{i,kj^{-1}} + \delta V_{j,i^{-1}k} = \delta V_{ij,k}$$

\begin{tabular}{c|ccc}
Case & $i$ & $j$ & $k$ \\ \hline \\
1 & $A$ & $A$ & $A$ \\
2 & $A$ & $A$ & $B$ \\
3 & $A$ & $B$ & $A$ \\
4 & $A$ & $B$ & $B$ \\
5 & $B$ & $A$ & $B$ \\
6 & $B$ & $A$ & $B$ \\
7 & $B$ & $B$ & $A$ \\
8 & $B$ & $B$ & $B$ 
\end{tabular}

\par
For Case 1, all the $\delta V_{x,y}$ have $x$ and $y$ in subset $A$, which is
an abelian subgroup and for which all the $\delta V$ $V1$ entries are
zero.  

\begin{tabular}{c|ccc}
Case & $\delta V_{i,kj^{-1}}$ & $\delta V_{j,i^{-1}k}$ & $\delta V_{ij,k}$ \\ \hline \\
2 & $V2$ & $V2$ & $V2$ \\
3 & $V2$ & $V3$ & $V3$ \\
4 & $V1=0$ & $V4$ & $V4$ \\
5 & $V3$ & $V2$ & $V3$ \\ 
6 & $V4$ & $V1=0$ & $V4$ \\ 
7 & $V4$ & $V4$ & $V1=0$ \\ 
8 & $V3$ & $V3$ & $V2$ 
\end{tabular}

\par $V2$ and $V3$ are, of course intimately linked, and $V4$ is not related to 
either of them.

\par Using the notation described above, if $i$ and $j$ are in $A$ their combination
$ij$ is $mod(i+j)_p$.  For simplicity I will denote $mod(i+j)_p$ as $\{i+j\}$.  If 
$i \in A$, $i^{-1}$
we can represent as $N-i$ or simply $\{-i\}$ for convenience.  If $i \in B$, $i^{-1} = i$ as
each element is its own inverse.
When $i,j \in B$ (i.e. $i$ is really $mod(i)_N +N \equiv \{i\}+p$),
their combination $ij$ is $\{j-i\}+p$.

\begin{tabular}{cc|c}
$i \in$  & $j \in$ & $ij$ \\ \hline \\
$A$ & $A$ & $\{i+j\} \in A$ \\
$A$ & $B$ & $\{j-i\} +p \in B$ \\
$B$ & $A$ & $\{i-j\} +p \in B$ \\
$B$ & $B$ & $\{j-i\} \in A$
\end{tabular}


\subsection{V4}
\subsubsection{Case 7}
In this case $i \in B$, $j \in B$, and $k \in A$.  Since $ij \in A$, the right-hand
side of this equation is zero.
$$\delta V_{i,kj^{-1}} + \delta V_{j,i^{-1}k} = \delta V_{ij,k} =0$$


$$\delta V_{i,kj^{-1}} = \delta V_{i,\{j-k\}+p} = \delta V_{\{i\}+p,\{j-k\}+p}$$
$$\delta V_{j,ik} = \delta V_{j,\{i+k\}+p} = \delta V_{\{j\}+p,\{i+k\}+p}$$

\par The subscripts are hard to read, so define 
$$\delta V_{x,y} \equiv (( x, y ))$$
and we have 
\begin{equation}
(( \{i\}+p, \{j-k\}+p )) = - (( \{j\}+p, \{i+k\}+p ))
\end{equation}
\par We already know we're in the $V4$ block, so we can omit the $+p$ for clarity here.
$$(( \{i\}, \{j-k\})) = - (( \{j\}, \{i+k\}))$$

\par We can use this more easily if we define $\{J\} \equiv \{j - k\}$, which gives use
\begin{equation}\label{eq:partialdiagonal}
(( \{i\}, \{J\})) = - (( \{J+k\}, \{i+k\})) \forall k
\end{equation}

\par  Wrap-around diagonals have the same value.  For example, if $\{i\}=2$ and
$\{J\} =1$ has $(( \{1\}, \{1\} )) = -x$, then in $V4$ we have:

\begin{equation}
\label{eq:diagonal}
\begin{tabular}{ccccccc}
 0 & x & 0 & 0 & \dots & 0 & -x\\
-x & 0 & x & 0 & \dots & 0 & 0\\
0 & -x & 0 & x & \dots & 0 & 0\\
0 & 0 & -x & 0 & \dots & 0 & 0\\
 &&&& \dots & & \\
0 & 0 & 0 & 0 & \dots  & 0 & x \\
x & 0 & 0 & 0 & \dots  & -x & 0
\end{tabular}
\end{equation}

There are $p$ partial diagonals, which are paired.  The central diagonal is zero.
If $p$ is odd there can be $(p-1)/2$ of these ($b^p=0$) set of diagonals.  If $p$ is even,
then we can use Equation~\ref{eq:partialdiagonal} and set $i=0$, $J=p/2$, and use
$k=p/2$ to find $(( 0, p/2)) = - (( 0, p/2))$.  Clearly all elements in this partial diagonal
are also 0.  So if $p$ is even the number of paired diagonals is $(p-2)/2$ and if $p$
is odd there are $(p-1)/2$.

\subsubsection{Case 6}
In this case $i \in B$, $j \in A$, and $k \in B$.  Since $j$ and $i^{-1}k$ 
are both in $A$, the
second term is 0 and the fundamental equation is now:
$$\delta V_{i,kj^{-1}}  + 0 = \delta V_{ij,k}$$

Once again this involves elements of $V4$
$$(( \{i\}+p, \{k-j\} +p )) = (( \{i+j\}+p, \{k\}+p))$$

If we drop the $p$ for clarity, and define $\{K\} = \{k-j\}$ then we have
$$(( \{i\}, \{K\} )) = (( \{i+j\}, \{K+j\})) \forall j$$

This simply tells us that all elements in a wrap-around diagonal are the 
same in $V4$, which we knew already from Case 7.

\subsubsection{Case 4}
In this case $i \in A$, $j \in B$, and $k \in B$.  Since both $i$ and $kj^{-1}$ are in $A$,
the first term is 0 and the fundamental equation is now:
$$0 + \delta V_{j,i^{-1}k} = \delta V_{ij,k}$$

This also involves elements of $V4$:
$$(( \{j\} +p, \{i+k\}+p )) = (( \{j-i\}+p, \{k\}+p))$$

If we drop the $p$ for clarity, and define $\{J\} = \{j-i\}$, then we have
$$(( \{J+i\}, \{k+i\} )) = (( \{J\}, \{k\} ))$$

This simple tells us that all elements in a wrap-around diagonal are the same
in $V4$, which we knew already from Case 7 and Case 6.  Simple parameter
substitution shows these 3 sets of equations are equivalent.

\subsubsection{Calculations within V4}

Since Cases 4, 6, and 7 exhaust the equations that involve elements of $V4$, there are
no further restrictions on the number of independent solutions and we have
$(p-1)/2$ (or $(p-2)/2$ if $p$ is even) generators which can be represented by the following block array,
where $0$ is an $p\times p$ array of $0$ and $V4$ is a wrap-around diagonal
array like that in Eq~\ref{eq:diagonal}.  For purposes of categorization we can without
loss of generality set x=1.  The wrap-arounds start in positions
$((\{1\},\{2\}))$ (for +1) and $((\{2\},\{1\}))$ (for -1) for the first array, $((\{1\},\{3\}))$ (for +1) 
and $((\{3\},\{1\}))$ (for -1) for the second array, and
so on up to $((\{1\},\{p+1)/2\}))$ (for +1) for the $(p-1)/2)$'th array.  
Give the one with  $((\{1\},\{m+1\}))=1$ the name $G_m$.

This is easier to deal with if we define an array
\begin{equation}
P=
\begin{cases}
P_{i,j} = 1  \text{ for } j=i+1 \\
P_{i,j} = 1  \text{ for } i=p,j=1 \\
P_{i,j} = 0  \text{ otherwise}
\end{cases}
\end{equation}

We can define our $(p-1)/2)$ generator representations as
\begin{equation}
G_m = P^m - P^{p-m} \text{ for } m \in \{1 \dots (p-1)/2\}
\end{equation}

Obviously $G_0 = 0$, and if $p$ is even $G_{p/2}=0$ too.  From this construction it is easy to see that these generator representations always commute.
\begin{equation}
G_m G_n = G_n G_m
\end{equation}

It further follows (recalling that $P^p=I$), and considering $3m$ as being modulo $p$, that
\begin{eqnarray}
{G_m}^3 & =&  P^{3m} - 3 P^{2m} P^{p-m} + 3 P^m P^{2p-2m} - P^{3p-3m} \\
&=& (P^{3m} - P^{p-3m}) - 3 (P^m - P^{p-m}) \\
&=& G_{3m} - 3 G_m
\end{eqnarray}

If $p$ is odd, or if $p$ is even and $m\neq p/2$, $3m$ will always refer to a different generator representation
than $m$.  Note that if $p>3m>(p-1)/2$ then $G_{3m} = -G_{p-3m}$

\begin{eqnarray}
G_m G_n G_l = G_{m+n+l} + G_{m-n-l} + G_{l-m-n} + G_{n-m-l}
\end{eqnarray}

When $p=3$ or $p=4$, there is only one generator in the representation of $V4$.  When
$p=5$ there are two, and it is straightforward to design linear combinations $L_1$ and
$L_2$ such that ${L_1}^3 = \alpha L_1$ and ${L_2}^3 = \beta L_2$.  For $L_1 = c_1 G_1 
+ c_2 G_2$, the ratio  $c_1/c_2$ is one of $(1 \pm \sqrt(5))/2$ or $(-2 \pm \sqrt(5))$.

For calculations define
\begin{eqnarray}
H_m = P^m + P^{p-m}
\end{eqnarray}

\par $H_0 = 2 I$, $H_m=H_{-m}$.  Even products of $G$'s involve  only the $H$'s, and odd products only
involve the $G$'s.

\subsection{V2, V3}
From $\delta V_{i,j} = - \delta V_{j^{-1},i^{-1}}$ we know that these two sub-matrices
are not independent.  Cases 2, 3, and 5, when re-written to reflect only elements in $V2$,
are equivalent sets of equations.  For simplicity of notation if $k^{\prime}>p$ I use the 
$k \equiv k^{\prime} -p$, 
element $k^{\prime} = a b^k$ below.
$$\delta V_{b^i,ab^{k-j}} + \delta V_{b_j,ab^{k+i}} = \delta V_{b^{i+j},ab^k}$$
$$\delta V_{b^i,ab^{j-k}} - \delta V_{b^{i-k},ab^j} = - \delta V_{b^{-k}, ab^{j-i}}$$
$$-\delta V_{b^{j-k},ab^i} + \delta V_{b^j, ab^{k+i}} = -\delta V_{b^{-k},ab^{i+j}}$$

Keeping only the exponents of $b$ and taking the presence of $a$ in the second
place as given, I write the first equation (equivalent to the other two) as
\begin{equation} \label{eqn:case235}
\langle i, k-j\rangle + \langle j, k+i\rangle = \langle i+j, k\rangle
\end{equation}

Equation Case 8 is not equivalent to the other three (2, 3, and 5).  
When I put it in a form that references
only $V2$ I get
\begin{equation} \label{eqn:case8}
 \langle k-j, i\rangle +\langle i-k, j\rangle + \langle j-i, k\rangle = 0
\end{equation}
Here $i$, $j$, and $k$ range from $0$ to $p-1$

\par If you set $j=i$ in Equation~\ref{eqn:case8} you can easily see that
\begin{equation} \label{eqn:reflect}
\langle x, y \rangle = - \langle -x, y \rangle
\end{equation}

\par If $p$ is odd the above are all distinct except where $x=0$, and we know already that 
$\langle 0, y \rangle=0$.  If $p$ is even, then $x=p/2=-x$ and $\langle p/2, y \rangle =
-\langle p/2, y \rangle$ and these are also clearly $0$ for all $y$:  a second row of $0$.

\par It isn't hard to show that Equation~\ref{eqn:case8} is a consequence of
Equation~\ref{eqn:reflect} and Equation~\ref{eqn:case235}, so we use the latter 
two as a simpler pair.

\par It is convenient to use Equation~\ref{eqn:case235} in a slightly different form, defining
$H = i + j$ to get
\begin{equation} \label{eqn:hk}
\langle H, k \rangle = \langle i, k - H + i \rangle + \langle H - i, k + 1 \rangle
\end{equation}

\par Let $H = p-1$.  From Equation~\ref{eqn:reflect} if we know $\langle p-1, k \rangle$
we know $\langle 1, k \rangle$.  Since the sums of each row or column of the differential matrix
is $0$, we also have
\begin{equation}
\langle p-1, 0 \rangle = \sum_{j=1}^{p-1} \langle p-1, j \rangle 
\end{equation}
and using Equation~\ref{eqn:reflect} we find
\begin{equation}
\langle p-1, k \rangle = \langle i, k+i+1 \rangle - \langle i+1, k+1 \rangle
\end{equation}

\par Let $I=1$.
\begin{equation}
\langle p-1, k \rangle = \langle 1, k+2 \rangle - \langle 2, k+1 \rangle =
-\langle p-1, k+2 \rangle - \langle 2, k+1 \rangle
\end{equation}

We can rewrite this as
\begin{equation}
\langle 2, k+1 \rangle = - \langle p-1, k \rangle + \langle p-1, k+2 \rangle
\end{equation}

Since this is true for all $k$, we now have each element of the row $\langle 2,x \rangle$
In terms of the $p-1$ remaining elements of the row $\langle p-1, x \rangle$.  This
automatically gives us the row $\langle p-2, x \rangle$.

\par Let $I=2$.  The same sort of manipulation shows us
\begin{equation}
\langle p-1, k \rangle = \langle 2, k+3 \rangle - \langle 3, k+2 \rangle
\end{equation}

Since we know the $\langle 2, x \rangle$ in terms of the top row, we can solve
for $\langle 3, x \rangle$ also, and plainly iterate through each of the rows in turn.

\par All rows are expressible in terms of the elements of the top row, and one of 
those elements in the top row in turn is expressible in terms of the rest of the row, so there are a maximum
of $p-1$ independent variables if $p$ is odd, and therefore a maximum of $p-1$ generators
involving the off-diagonal arrays.

\par Explicit calculations on examples suggest that $p-1$ is also the minimum, which,
combined with the generators from the diagonal matrix, says there are $3 (p-1)/2$ 
generators for the transformations of a finite group of order $2p$.

\par We can produce neater equations.  Let $H=p-2$ and $i=p-1$.

\begin{equation}
\langle p-2, k \rangle = \langle p-1, k+1 \rangle + \langle p-1, k-1 \rangle
\end{equation}

Then, successively, let $H=p-3$ and $i=p-2$, then $H=p-4$ and $i=p-3$ to get
\begin{equation}
\langle p-3, k \rangle = \langle p-1, k+2 \rangle + \langle p-1, k \rangle + \langle p-1, k-2 \rangle
\end{equation}
\begin{equation}
\langle p-4, k \rangle = \langle p-1, k+3 \rangle + \langle p-1, k+1 \rangle +
\langle p-1, k-1 \rangle + \langle p-1, k-3 \rangle
\end{equation}

and so on.
\begin{equation} \label{eqn:struct}
\langle p-X, k \rangle = \sum_{a=1-X}^{X-1 , (2)} \langle p-1, k+a \rangle
\end{equation}

This makes it convenient to deal with the case where $p$ is even.  Recall that
$$\langle p/2, k \rangle=0$$.  In Equation~\ref{eqn:struct} set $X=p/2$.  This gives us
\begin{equation}
\langle p-p/2, k \rangle = \langle p/2, k \rangle = 0 = \sum_{a=1-p/2}^{p/2-1, (2)} \langle p-1, k+a \rangle
\end{equation}
The right hand side consists of $(p-2)/2 +1 =p/2$ terms, or half of the $\langle p-1,  k \rangle$ row.
If $k$ is odd these are all the odd numbered elements, and if $k$ is even all the even numbered
elements of the row.  This gives 2 equations:  all even numbered elements of the $p-1$ row
sum to zero, and so do all the odd numbered elements.  Adding both says that the sum of all
elements in the row is zero, which we already knew.  Therefore there are 2 independent
constraints on the values of the elements of the $p-1$ row (from which all other rows can
be determined), so there are $p-2$ independent parameters in the transformation over the 
off-diagonal sub-arrays.

\par Combining the diagonal sub-matrix results and the off-diagonal sub-matrix results tells us
that for a dihedral group of order $p$, there are a maximum of $3(p-1)/2$ independent
parameters in the transformation if $p$ is odd, and $3(p-2)/2$ if $p$ is even.

\par In all cases I have solved the maximum number of parameters is the total number of parameters.
\section{H a subgroup of G}

We use the differential continuous transformation equation
\begin{equation}
\delta V_{i,kj^{-1}} + \delta V_{j,i^{-1}k} = \delta V_{ij,k}
\end{equation}
\par For simplicity of notation, take the $\delta V$ as given and
represent the above as 
\begin{equation}
(i,kj^{-1})+ (j,i^{-1}k) = (ij,k)
\end{equation}
Let $T_H$ be the continuous transformations (defined in 
\footnote{Journal of Mathematical Physics 38:3414-3426, 1997}) over H. Let H be a subgroup of G.

\par Is $T_H \leq T_G$?

\subsection{H is a normal subgroup of G}

\subsubsection{$G = H \otimes Z_2$}

For this case, $T_H \leq T_G$, as I show below.

\par The cosets of $G$ are $H$ and $Q$, where $Q=Hb$ and $b^2 =0$, with $0$ representing the identity.
Retain the ordering, so that each element $h_b$ of $Q$ is $h b$ where $h \in H$.  Since $b$ commutes
with everything I will place it at the right.

\par The $i$, $j$, and $k$ will be in either $H$ or $Q$, giving us 8 sets of equations.  The transformation
array partitions into 4 blocks.  $A$ and $D$ are on the diagonal, $B$ and $C$ off-diagonal.
$$
\begin{array}{ccc}
 & H & Q \\
H & A & B \\
Q & C & D
\end{array}
$$

\par The list of equations is
$$
\begin{array}{ccccccc}
  & i & j & k & Generic & Block & Detail \\
1 & H & H & H & (H,H)+(H,H)=(H,H) & A+A \in A & (i,kj^{-1}) + (j,i^{-1}k)=(ij,k) \\
2 & H & H & Q & (H,Q)+(H,Q)=(H,Q) & B+B \in B & (i,kj^{-1}b) + (j,i^{-1}kb)=(ij,kb) \\
3 & H & Q & H & (H,Q)+(Q,H)=(Q,H) & B+C \in C & (i,kj^{-1}b) + (jb,i^{-1}k)=(ijb,k) \\
4 & H & Q & Q & (H,H)+(Q,Q)=(Q,Q) & A+D \in D & (i,kj^{-1}) + (jb,i^{-1}kb)=(ijb,kb) \\
5 & Q & H & H & (Q,H)+(H,Q)=(Q,H) & C+B \in C & (ib,kj^{-1}) + (j,i^{-1}kb)=(ijb,k) \\
6 & Q & H & Q & (Q,Q)+(H,H)=(Q,Q) & D+A \in D & (ib,kj^{-1}b) + (j,i^{-1}k)=(ijb,kb) \\
7 & Q & Q & H & (Q,Q)+(Q,Q)=(H,H) & D+D \in A & (ib,kj^{-1}b) + (jb,i^{-1}kb)=(ij,k) \\
8 & Q & Q & Q & (Q,H)+(Q,H)=(H,Q) & C+C \in B & (ib,kj^{-1}) + (jb,i^{-1}k)=(ij,kb) 
\end{array}
$$
\par Notice that equation sets 1, 4, 6, and 7 include only $A$ and $D$, and the rest only include
$B$ and $C$.  Equation set 1, if in isolation, would reproduce $T_H$.  Since 2, 3, 5, and 8
do not involve these elements, it suffices to show that equations 1, 4, 6, and 7 will in fact
continue to reproduce $T_H$.

\par Add up equation sets 4, 6, and 7.  This gives
\begin{equation}
(i,kj^{-1}) + (j,i^{-1}k) + 2 (jb,i^{-1}kb) + 2 (ib,kj^{-1}b) = (ij,k) + 2 (ijb,kb)
\end{equation}
which of course simplifies to
\begin{equation}
(jb,i^{-1}kb) + (ib,kj^{-1}b) = (ijb,kb)
\end{equation}

\par Add equation sets 4 and 6.  This gives
\begin{equation}
(i,kj^{-1}) + (jb,i^{-1}kb) + (ib,kj^{-1}b) + (j,i^{-1}k) = 2 (ijb,kb)
\end{equation}
which we combine with the above equation to find that
\begin{equation}
(i,kj^{-1}) + (j,i^{-1}k) = (ijb,kb)
\end{equation}

\par Solving for the rest of them is trivial, and we see that
\begin{equation}
(ijb,kb) = (ij,k) \quad (ib,kj^{-1}b) = (i,kj^{-1}) \quad (jb,i^{-1}kb) = (j,i^{-1}k)
\end{equation}

In this case we see that the transformations within $D$ exactly mirror those within $A$, which
means that the same $T_H$, extended to reach the elements in $Q$, is part of the set of transformations
over $H \otimes Z_2$.

\par For the sake of completeness, recalling that $(x,y) = - (y^{-1},x^{-1})$, we can show that
the off-diagonal blocks are governed by
\begin{align}
-(jk^{-1}b,i^{-1}) = (i,kj^{-1}b) &= (ib, kj^{-1}) = - (jk^{-1},i^{-1}b) \\
-(k^{-1}b,j^{-1}i^{-1}) = (ij, kb) &= (ijb, k) = - (k^{-1},j^{-1}i^{-1}b) \\
-(k^{-1}ib, j^{-1}) = (j,i^{-1}kb) &= (jb,i^{-1}k) = - (k^{-1}i, j^{-1}b)
\end{align}

\subsubsection{$H \otimes g$}

Order the elements of $g$ in some arbitrary order that has the identity as $0$.  Call the $q$'th element $g_q$.

The group $G$ partitions into cosets which we can label as $C_q = H g_q$.

Recall the fundamental equation:
\begin{equation}
(i,kj^{-1})+ (j,i^{-1}k) = (ij,k)
\end{equation}

If the $\{i,j,k\} \in H$, we have the transformation $T_H$ over $H$, provided it is
not constrained by any other terms.

In the following cases, the $\{i,j,k\} \in H$ is already accounted for, and $q \ne 0$.

If the first term involves elements of $(H,H)$, then we have $i \in H$, and if $k \in C_q$, then
we must have $j \in C_q$.  This means that the second and third terms must be in $(C_q,C_q)$
and $(C_q,C_q)$ respectively.  $(H,H) + (C_q,C_q) \in (C_q,C_q)$.

If the second term involves elements of $(H,H)$, then we have $j \in H$, and if $k \in C_q$, then
we must have $i \in C_q$.  This means that the first and third terms must be in $(C_q,C_q)$ and
$(C_q,C_q)$ respectively.   $(C_q,C_q) + (H,H)  \in (C_q,C_q)$.

If the third term involves elements of $(H,H)$, then we have $k \in H$, and if $j \in C_q$, then
we must have $i \in C_q^{-1}$. This means that the first and second terms must be in $(C_q^{-1},C_q^{-1})$
and $(C_q^{-1},C_q^{-1})$ respectively.  $(C_q^{-1},C_q^{-1}) + (C_q^{-1},C_q^{-1}) \in (H,H)$.
Recalling that $(i,j) = -(j^{-1},i^{-1})$, we can have all these equations involving $(C_q,C_q)$, which
simplifies the argument below.

We have $1+3 \times (n-1)$ equation sets:  One that mixes only $(H,H)$ terms, and $(n-1)$ sets of $3$
equation sets mixing only $(H,H)$ and $(C_q,C_q)$ terms for a given $q$.  

\par However, though each of the of the $(n-1)$ sets of
equation sets is identical in form to each other, they are not as simple as the $Z_2$ example above.
Since $q^2 \neq 0$ in general, though the first two equation sets are the same, the last, in order to
involve only $C_q$ elements, reverses the order.
\begin{align}
 (i,kj^{-1}) + (jq,i^{-1}kq) & = (ijq,kq) \\ 
(iq,kj^{-1}q) + (j,i^{-1}k) &= (ijq,kq) \\ 
(iq,kj^{-1}q) + (jq^{-1},i^{-1}kq^{-1}) &= (ij,k) \nonumber \\ 
(iq,kj^{-1}q) - (k^{-1}iq,j^{-1}q) &= (ij,k) 
\end{align}

Combining all three reduces to the following, but without more information we cannot solve for
the $(C_q,C_q)$ terms in terms of the $(H,H)$ ones, and in general the interference between terms
will mean that the $T_H$ will no longer be a subset of $T_G$.
\begin{equation}
2(iq,kj^{-1}q) + (jq,i^{-1}kq) - (k^{-1}iq,j^{-1}q) = 2(ijq,kq)
\end{equation}

\subsubsection{Normal subgroup}

The mixing is even more thorough if H is a generic normal subgroup of G.  The $C_q$ we can still
define similarly, with $\{q\}$ being one element of each coset, but since in general $iq \neq qi$,
the sets of equation sets are more complicated:
\begin{align}
(i,kj^{-1}) + (jq, i^{-1}kq) &= (ijq, kq) \\ 
(iq,kqj^{-1}) + (j, i^{-1}k) &= (iqj, kq) \\
(iq,kqj^{-1}) + (jq^{-1}, q^{-1}i^{-1}k) &= (iqjq^{-1},k) \nonumber \\
(iq,kqj^{-1}) - (k^{-1}iq,qj^{-1}) &= (iqjq^{-1},k)
\end{align}

Clearly the equation sets for simple groups will be even further entangled, and $T_H$ will not
be a subset of $T_G$ except by accident.

\subsubsection{Consequence}

Taking this in reverse, we see that the transformations of $H$ when it is a subset of $G$ will
not in general be the same as the transformations of $H$ taken standalone, because of the
extra constraints due to its embedding in $G$.

\par This means that you cannot simply solve a larger group and automatically
retrieve the transformations of its subgroups as they would be in isolation:  each group
needs to be studied on its own.

\section{Physics applications}

Can the symmetries of this type have physics applications?

\par If the group in question is a representation of physical
entities whose interactions are faithfully represented by the
group's Cayley table, then yes, it could.  The original JMP
paper studied the possibility that preon interactions could be
modeled this way.

\par  If, on the other hand, the group is the representation of the
set of operators
on some physical system, then the answer is no, unless the
representation uses some other field than the field of which
the transformations are a subset.  This is because
the set of operators which preserves the symmetries of that
system will not include the {\it additions} (in the set of transformations) 
of two operators.  This {\it addition} 
creates a new operator which is not in the original set of 
symmetry-preserving operators.  For example, in the usual
representation, group element interactions are matrix multiplications,
but when you treat them as basis vectors for a field, that introduces matrix
addition, which creates matrices which don't have the same 
properties as the original ones.

\par In one simple case, an equilateral
triangle can be operated on by 3 flips $\{ F_1, F_2, F_3 \}$
and 3 rotations $\{ R_0, R_{120}, R_{240} \}$.  A set of 3 2-points
representing the triangle is operated on by $2x2$ matrices
representing the operators.  One infinitesimal transformation over
$F_1$ turns out to be $F_1 + \epsilon F_2 - \epsilon F_3$.  You
quickly see that the result doesn't preserve any of the points, or
transform them into each other.  This comes from mixing two
distinct operations:  transformations over the physical system and
transformations over the representation.

\par In addition, there are two distinct entities which can represent
null.  The identity element, modulo some scaling, is one candidate,
but so is the true zero, in which all coefficients of the group elements
are zero.  Notice that for two elements $a$ and $b$ such that $a^2=b^2$
and $ab=ba$, the expanded product of $a+b$ and $a-b$ is true zero,
even though neither of the terms is.

\section{ Examples}
As described in the earlier paper, the continuous transformations over
the dihedral group of order 6 (the smallest non-abelian finite group) are
isometric to SU(2).  The transformations over the alternating group $A_4$
(a 12-element group) are isometric to S(3).  

\par A tool for generating the commutation relations for small groups
may be found at GitHub:  https://github.com/jnbellinger/GroupAlgebraSymmetry.git


\end{document}
